
\begin{figure}
    \centering
    
    \subfloat[SF3K-Q1]{
        \includegraphics[scale=0.33, page=1]{expfig/vsgm-3k-q1-crop.pdf}
        \label{fig:vsgm_3kq1}
    }\hfil
    \subfloat[SF3K-Q2]{
        \includegraphics[scale=0.33, page=1]{expfig/vsgm-3k-q2-crop.pdf}
        \label{fig:vsgm_3kq2}
    }\hfil
    \subfloat[SF3K-Q3]{
        \includegraphics[scale=0.33, page=1]{expfig/vsgm-3k-q3-crop.pdf}
        \label{fig:vsgm_3kq3}
    } \\
     \vspace{-0.5em}
    \subfloat[SF10K-Q1]{
        \includegraphics[scale=0.33, page=1]{expfig/vsgm-10k-q1-crop.pdf}
        \label{fig:vsgm_10kq1}
    }\hfil
    \subfloat[SF10K-Q2]{
        \includegraphics[scale=0.33, page=1]{expfig/vsgm-10k-q2-crop.pdf}
        \label{fig:10kq2}
    }\hfil
    \subfloat[SF10K-Q3]{
        \includegraphics[scale=0.33, page=1]{expfig/vsgm-10k-q3-crop.pdf}
        \label{fig:10kq3}
    }
    \caption{Breakdown execution time  of VSGM and GCSM. `DC' includes the time for identifying the caching vertices and copying their neighbor lists to GPU. `Match' is the matching kernel execution time on GPU. }
    % \vspace{-1cm}
    \label{fig:vsgm}
\end{figure}