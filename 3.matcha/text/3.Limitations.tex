\section{Limitations with Existing Subgraph Matching Systems}




While numerous systems for subgraph matching have been proposed~\cite{mawhirter2019automine, mawhirter2021dryadic,wei2022stmatch,yuan2023everest,chen2022decomine,pinar2017escape,sun2020rapidmatch,aberger2017emptyheaded}, each of these systems features unique algorithms and optimizations. This makes it challenging to compare and combine  existing techniques, and to develop new techniques based on prior work. 

\noindent
\textbf{\textit{Restricted Algorithm Support.}} Previous subgraph matching systems implement algorithms that are best suited for specific tasks. For instance, DecoMine~\cite{chen2022decomine} employs a pattern-decomposition-based algorithm~\cite{pinar2017escape}; it is most efficient for counting subgraphs that can be divided into independent sub-patterns. RapidMatch~\cite{sun2020rapidmatch} implements a join-based algorithm; it works best for labeled queries. AutoMine~\cite{mawhirter2019automine} supports both labeled and unlabeled queries, but it only runs the basic backtracking algorithm that matches the pattern vertices one by one. 
We summarize the algorithm support of state-of-the-art subgraph matching systems in Table~\ref{tab:intro}.
%\textcolor{red}{Although systems like Peregrine~\cite{jamshidi2020peregrine} and G2Miner~\cite{chen2022efficient} claim to be general-purpose, they actually hard-code matching algorithms for specific patterns. }
%It cannot be used to implement algorithms other than those provided. 
%Also, these systems cannot be easily extended to support new algorithms. 
%For example, one might consider combining RapidMatch and Decomine by using RapidMatch to match each sub-pattern from Decomine. However, modifying either Decomine or RapidMatch to implement the algorithm is challenging. 

\noindent
\textbf{\textit{Inconsistent Optimizations.}} The existing subgraph matching systems are inconsistent in performance optimizations.  First, they are implemented on different platforms. Most of the systems are CPU-based~\cite{sun2020rapidmatch,10.14778/1453856.1453899,10.1145/3299869.3319880,mawhirter2019automine,mawhirter2021dryadic,chen2022decomine}, while some recent systems utilize accelerators such as GPUs to speed up computation~\cite{zeng2020gsi,xiang2021cuts,wei2022stmatch,chen2022efficient}. This variety makes it difficult to compare and combine optimization techniques across different platforms. For instance, RapidMatch~\cite{sun2020rapidmatch} is a CPU-only system. It is unclear whether its join-based algorithm can run efficiently on GPU and whether it can outperform the optimized GPU systems such as STMatch~\cite{wei2022stmatch} and G2Miner~\cite{chen2022efficient} that use more basic vertex-extension algorithms. 
%DecoMine~\cite{chen2022decomine} is another system that only runs on CPU. Migrating it to GPU will require a significant amount of engineering work. 
Even on the same hardware, different systems feature different optimizations, as summarized in Table~\ref{tab:intro}. 

%For example, both G2Miner~\cite{chen2022efficient} and STMatch~\cite{wei2022stmatch} implement the vertex-extension backtracking algorithm on GPU. G2Miner employs a pattern-merging strategy to eliminate redundant computation for matching multiple patterns. On the other hand, STMatch may achieve better performance for matching a single pattern due to its work-stealing technique. Combining these two optimizations within either system is challenging. 


% Table generated by Excel2LaTeX from sheet 'Sheet2'

%Table~\ref{tab:intro} summarizes the features of the most recent subgraph matching systems. 
Given the limitations of the existing systems, a unified framework that allows the development and integration of subgraph matching algorithms and optimization techniques across different platforms is desirable.