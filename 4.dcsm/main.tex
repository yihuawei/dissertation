\documentclass[sigconf]{acmart}

\usepackage{amsmath, amsfonts}
\usepackage{graphicx}
\usepackage{textcomp}
\usepackage{xcolor}
\usepackage[linesnumbered,ruled,vlined]{algorithm2e}
\usepackage{wrapfig}
\usepackage{tikz}
\usepackage{subfig}
\usepackage{url}
\usepackage{ragged2e}
\usepackage{booktabs,makecell, multirow, tabularx}
\usepackage{bigstrut}
\usepackage{multirow}
\usepackage{amsthm}
\usepackage{enumitem}
\usepackage{circledsteps}
\usepackage{listings}
\usepackage{soul} 
\usepackage{color, xcolor}
\usepackage[most]{tcolorbox}
\usepackage[10pt]{moresize}
\usepackage{pifont}
\usepackage{rotating}

\AtBeginDocument{%
  \providecommand\BibTeX{{%
    Bib\TeX}}}


\setcopyright{acmlicensed}
\copyrightyear{2018}
\acmYear{2018}
\acmDOI{XXXXXXX.XXXXXXX}
%% These commands are for a PROCEEDINGS abstract or paper.
\acmConference[Conference acronym 'XX]{Make sure to enter the correct
    conference title from your rights confirmation email}{June 03--05,
    2018}{Woodstock, NY}
%%
%%  Uncomment \acmBooktitle if the title of the proceedings is different
%%  from ``Proceedings of ...''!
%%
%%\acmBooktitle{Woodstock '18: ACM Symposium on Neural Gaze Detection,
%%  June 03--05, 2018, Woodstock, NY}
\acmISBN{978-1-4503-XXXX-X/2018/06}


\settopmatter{printfolios=true}
\settopmatter{printacmref=false}
%\pagestyle{plain}
%https://arxiv.org/pdf/2412.06993
%%
%% end of the preamble, start of the body of the document source.
\begin{document}


\title{DCSM: Enabling Inter-Batch Parallelism for Continuous Subgraph Matching on GPU}
% \subtitle{\normalsize{ICS 2025 Submission
%     \textbf{\#NaN} -- Confidential Draft -- Do NOT Distribute!!}}

% \tcolorboxenvironment{lstlisting}{
%   spartan,
%   frame empty,
%   boxsep=0mm,
%   left=1mm,right=1mm,top=-1mm,bottom=-1mm,
%   colback=gray!45,
% }

\newtheorem{property}{Property}

\definecolor{block-gray}{gray}{0.9}
\newtcolorbox{myquote}{colback=block-gray, boxrule=0pt,boxsep=0pt, breakable}

\definecolor{codegreen}{rgb}{0,0.6,0}
\definecolor{codegray}{rgb}{0.5,0.5,0.5}
\definecolor{codepurple}{rgb}{0.58,0,0.82}
\definecolor{backcolour}{rgb}{0.95,0.95,0.92}

\lstdefinestyle{mystyle}{
    commentstyle=\color{codegreen},
    keywordstyle=\color{blue}, 
    numberstyle=\color{codegray},
    stringstyle=\color{codepurple},
    basicstyle=\small\ttfamily,       
    breakatwhitespace=false,         
    breaklines=true,                 
    captionpos=b,                    
    keepspaces=true,                                 
    numbersep=5pt,                  
    showspaces=false,                
    showstringspaces=false,
    showtabs=false,                  
    tabsize=4,
    mathescape=true
}
\lstset{style=mystyle}


\author{anonymous authors}
% \affiliation{%
%   \institution{Anonymous Institution}
% }

\begin{abstract}
Continuous subgraph matching (CSM) is a fundamental building block in many real-world applications. 
While prior studies have explored executing CSM on heterogeneous systems with GPUs, they only exploit intra-batch parallelism and cannot process multiple batches concurrently—a capability essential for handling real-time requests. 
In this work, we propose a GPU-based system to accelerate CSM in practical, real-world settings. 
We adopt an algorithm-system co-design approach to unlock inter-batch parallelism. 
We introduce several key components, including a warp-specialized execution model and a multi-version graph data structure, along with version control logic for CSM tasks. 
Additionally, we propose optimizations such as warp-level parallel execution for data copying and incremental matching. 
Experimental results show that our system demonstrates optimal throughput and response time on GPU platforms across various update arrival rates.

\end{abstract}

% \keywords{Do, Not, Us, This, Code, Put, the, Correct, Terms, for, Your, Paper}


\maketitle


\section{Introduction}
\label{sec:Intro}
\textit{Continuous subgraph matching} (CSM) refers to the process of incrementally matching a query pattern against a dynamic data graph. 
CSM plays a critical role in many real-world graph analytics applications. 
For example, transactions in e-commerce platforms can be modeled as a dynamic graph, and CSM can be employed to detect fraudulent merchants \cite{qin2019towards}. 
It can also be applied to monitor money laundering in transaction networks \cite{qiu2018real}, trace rumor propagation paths in social networks \cite{wang2015detecting}, and identify system anomalies in computer communication networks \cite{manzoor2016fast}.

Figure~\ref{fig:csm} illustrates an example of CSM.
Given an initial data graph $G_0$ at time $t=0$ and a query pattern $Q_d$, $G_0$ contains one matching subgraph $(v_0, v_2, v_3, v_5)$.
At $t=3$, an edge $(v_1, v_3)$ is added to $G_0$, the updated graph $G_1$ produces an additional match $(v_0, v_1, v_2, v_3)$ for the query. 
At $t=5$, the edge $(v_3, v_5)$ is deleted from the graph, which invalidates the previous match $(v_0, v_2, v_3, v_5)$. 

\begin{figure}[ht]
    \centering
    % \captionsetup[subfigure]{width=75pt}%
    \subfloat[Query $Q_{d}$]{
        \includegraphics[scale=0.32, page=1]{./fig/slides-crop.pdf}
        \label{fig:query}
    } \hfil
    \hspace{2em}
    \captionsetup[subfigure]{width=75pt}%
    \subfloat[$t$=$0$: Data graph $G_0$]{
        \includegraphics[scale=0.32, page=2]{./fig/slides-crop.pdf}
        \label{fig:g0}
    } \\
    \captionsetup[subfigure]{width=75pt}%
    \subfloat[$t$=$3$: Data graph $G_1$]{
        \includegraphics[scale=0.32, page=3]{./fig/slides-crop.pdf}
        \label{fig:g1}
    } \hfil
    % \hspace{-3em}
    \captionsetup[subfigure]{width=75pt}%
    \subfloat[$t$=$5$: Data graph $G_2$]{
        \includegraphics[scale=0.32, page=4]{./fig/slides-crop.pdf}
        \label{fig:g1}
    }
    \caption{An example of continuous subgraph matching. $G_{k+1}$ is the data graph after applyting update on $G_{k}$. $Q_{d}$ is the query of "diamond" structure. $abcd$ next to the vertices are the vertex labels.}
    \label{fig:csm}
    %\vspace{-1em}
\end{figure}

Many efforts have been made to improve the performance of CSM on CPUs, primarily by reducing the search space during the matching process~\cite{kim2018turboflux, min2021symmetric, sun2022rapidflow, li2024newsp}.
However, due to the exponential time complexity of subgraph matching~\cite{ullmann1976algorithm}, CSM remains a computationally expensive procedure.
To address this limitation, recent systems \cite{wei2024gcsm, qiu2024gpu} have begun exploiting the massive parallelism of GPUs to accelerate CSM. 

The existing GPU-based CSM systems process updates using fixed-size batches.
Specifically, they wait until a predefined number of edge updates accumulate before grouping them into a batch and launching it on the GPU.
To fully utilize GPU cores, the batch size is typically set to a large value (e.g., 4096).
While a large batch size improves the GPU occupancy during the matching procedure, it increases the response time of individual updates and may even reduce overall GPU utilization, since edges must wait for the entire batch to be formed before execution. 

To validate this point, we tested the state-of-the-art GPU-based CSM system (GCSM~\cite{wei2024gcsm}) under different batch sizes. 
The performance results are presented in Figure~\ref{fig:rtime}. We define the response time of a graph update as the interval between its arrival and the start of its processing. 
We observe that GCSM exhibits poor response time under different graph update rates. 
With a batch size of 4096, GCSM-4096 suffers from long response times at low update rates, as it must wait to accumulate a sufficient number of updates to form a large batch. 
Conversely, with a batch size of 128, GCSM-128 experiences longer response times at high update rates due to its limited parallelism and thus low GPU utilization, as shown in Figure~\ref{fig:gutil}. 

% \textcolor{blue}{ } 

\begin{figure}[h]
    \centering
    \subfloat[Response Time]{
        \includegraphics[scale=0.42, page=1]{./fig/profile-crop.pdf}
        \label{fig:rtime}
    } \hfil
    \subfloat[GPU Utilization]{
        \includegraphics[scale=0.42, page=1]{./fig/profile2-crop.pdf}
        \label{fig:gutil}
    } 
    \caption{Average response time and GPU utilization of GCSM. GCSM-x denotes GCSM processing with batch size x. \textbf{Query:} Q4 (Figure~\ref{fig:queries}). \textbf{Graph:} Unlabeled Netflow~\cite{netflow}.}
    \label{fig:profile}
\end{figure}


To address the limitation of existing GPU-based CSM systems, we propose DCSM in this work. 
We find that the problem with existing systems stems from the fact that they cannot process different batches in parallel. 
A batch must wait for the previous batch to finish processing before it can be launched. 
DCSM addresses the response time and utilization issue by enabling parallel execution across batches. 
Intuitively, such inter-batch parallelism allows our system to process graph updates in small batches while keeping a high GPU occupancy. 
As shown in Figure~\ref{fig:profile}, our system maintains short response times across a wide range of update rates and achieves higher average GPU utilization than GCSM. 


The key technique that enables inter-batch parallelism in our system is resolving data races on the data graph while processing multiple batches concurrently.
We propose the first {\bf multi-version graph data structure} for CSM. It records a distinct graph version with minimum overhead for each batch update, allowing safe and efficient inter-batch parallelism. 
To manage memory consumption, we develop a {\bf garbage collection mechanism} that releases graph versions no longer in use.
We further introduce several system-level optimizations, including {\bf pipelined and parallel updates} to the multi-version graph using GPU warps, speeding up the creation of new graph versions, and a {\bf dynamic scheduling strategy} that assigns pending batches to idle warps to achieve better load balance among warps.

% These techniques help our system achieve
% \textcolor{red}{These techniques help our system achieve....}
%\textcolor{blue}{an efficient neighbor accessing strategy for multi-version graphs, and a garbage collector implementation on GPU.}
% \textcolor{red}{To ensure the correctness of the matching on this new multi-version graph, we provide a formal proof. we don't need this in the intro..}


We evaluate our system against state-of-the-art CPU and GPU systems. 
The experiments demonstrate that DCSM ensures optimal response time and throughput across different update rates. 
Compared to GCSM~\cite{wei2024gcsm}, DCSM achieves up to 87.9x speedup under high update rates and up to 312x speedup under low update rates. 
Furthermore, DCSM achieves 10.5x higher throughput compared to the state-of-the-art CPU-based CSM system RapidFlow~\cite{sun2022rapidflow}.


%
% \textcolor{blue}{Under high arrival rate, DCSM achieves an average 60x speedup in response time compared to GCSM-128. Under low arrival rate, DCSM achieves an average 21x speedup compared to GCSM-4096. } 

% 要测试测试GCSM不同的batch size
% 不同的arrival rate下,不同的batch size下,都比GCSM更好
% Most case下都比GCSM好,极端case下和GCSM一样、

% \begin{itemize}[leftmargin=0.3cm, itemindent=0cm]
% \item \textcolor{green}{It's better to test with different batch sizes for all graph update rates, say something like: no matter what batch size is used for GCSM, we outperform the best GCSM configuration across a wide range of graph update rates...also, add some CPU comparison results.. }
% \item \textcolor{green}{can we change the system to a better name? reflecting the inter-batch parallelism design?}
% \item \textcolor{green}{I removed the contribution bullets, as they are duplicate to this paragraph, you may add a little bit more details in the paragraph. }
% \end{itemize}





\iffalse
In summary, the main contributions of this work are as follows:
\noindent
\begin{itemize}[leftmargin=0.3cm, itemindent=0cm]
\item We design a strategy to enable inter-batch parallelism, which provides flexibility in handling varying update arrival rates.
\item We design an efficient system on GPU and address various performance issues, such as memory access inefficiency and load imbalance, thereby fully utilizing GPU resources.
\item Through the overall system design, we ensure that our system achieves good throughput, response time, and GPU utilization, enabling it to flexibly adapt to real-world demands.
\end{itemize}
\fi
% \textcolor{green}{Problems resolved}
% \noindent
% \begin{itemize}[leftmargin=0.3cm, itemindent=0cm]
% \item  \textcolor{green}{I think we want to show small batch size leads to shorter response time, right?}
% \item  \textcolor{green}{describe our results...} 
% \item  \textcolor{green}{briefly describe the techniques you used, 3-5 sentences..}
% \end{itemize}
\iffalse
\textcolor{green}{Problems resolved}
\noindent
\begin{itemize}[leftmargin=0.3cm, itemindent=0cm]
\item \textcolor{green}{why $Q_d$? change it $Q$?}
\item \textcolor{green}{this paragraph needs to be rewritten...give more details, explain why inter-batch parallelism can address the response time and utilization problem...}
\item \textcolor{green}{what is GCSM-6000?} 
\item \textcolor{green}{using what query pattern on what data graph, give reference..remove Rapidflow data...} 
\item \textcolor{green}{explain response time, GCSM-1000...each figure should have three lines: GCSM-128, GCSM-4096, Our}
\item \textcolor{green}{add some highlights of experimental results, for example, compared to what system, the response time is reduced by xxx, the overall query processing time is reduced by xxx.}
\end{itemize}
\fi
%GCSM-4096 exhibits localized valleys in GPU utilization, as the GPU remains idle during batch formation periods. GCSM-128 maintains consistently low GPU utilization due to insufficient parallelism with a batch size of 128. However, neither can fully utilize the GPU.}

% \textcolor{blue}{To validate this point, we tested the state-of-the-art GPU-based CSM system (GCSM~\cite{wei2024gcsm}) with different batch sizes for matching a 3-star pattern (Q4 in Fig.~\ref{fig:queries}) on the Netflow graph~\cite{netflow}. 
% The results are shown in Fig.~\ref{fig:rtime}. 
% We define the response time of a graph update as the time interval between its arrival and the start of its processing. 
% We observe that GCSM-4096's response time first decreases until the arrival rate reaches a threshold (12K updates/second), and then increases. Before the threshold, higher arrival rates shorten the time needed to form a batch, resulting in lower response times. After the threshold, arrival rates exceed GCSM-4096's maximum throughput, causing new batches to wait for previous ones to complete. GCSM-128 exhibits a similar trend, but with some differences. GCSM-128 performs better at low arrival rates as smaller batches form more quickly, but has significantly higher response times at high arrival rates because its low parallelism limits throughput. We also profile GCSM's GPU utilization, and show the results in Fig.~\ref{fig:gutil}. GCSM-4096 exhibits localized valleys in GPU utilization, as the GPU remains idle during batch formation periods. GCSM-128 maintains consistently low GPU utilization due to insufficient parallelism with a batch size of 128. However, neither can fully utilize the GPU.}

% We design a multi-version graph data structure to resolve update-match conflicts between batches. Our multi-version graph incurs negligible overhead compared to incremental matching. We also propose optimizations to the multi-version graph to improve GPU utilization. In summary, we make the following contributions:

%
%In some real-world scenarios, an update requires a quick response. For example, a merchant fraud detection system receives real-time transaction records and should alert the monitoring system as quickly as possible \cite{qin2019towards}. 


% 换一个名字,不要用DCSM

% 要测试测试GCSM不同的batch size
% 不同的arrival rate下,不同的batch size下,都比GCSM更好
% Most case下都比GCSM好,极端case下和GCSM一样、


%Fig4分成A B C三个子图, legend改成箭头,指向Form batch graph指向对应位置。





%俩typically

%4.3说的具体一些

% Property不能被proof。

% 不能有任何的proof

%EX加一张图。

% Property可以保留,用一个框或者图片来说明。
\section{Background}

\label{sec:Background}

 \subsection{A Formal Definition of CSM}
%This section gives formal definitions of continuous subgraph matching problem. 
\label{sec:Preliminaries}

\begin{definition}[Graph]
A graph $G$ = $(V, E, L)$, consists of a set of vertices $V$, a set of edges $E$, and a labeling function $L$ that assigns labels to vertices. 
\end{definition}

\begin{definition}[Subgraph]
A graph $G'$ = $(V', E', L')$ is a subgraph of graph $G = (V, E, L)$ if $V'$ is a subset of $V$, $E'$ is a subset of $E$, and $L'(v) = L(v)$, for all $v$ in $V'$. 
\end{definition} 

\begin{definition}[Isomorphism]
Two graphs $G_a$ = $(V_a, E_a, L_a)$ and $G_b$ = $(V_b, E_b, L_b)$ are isomorphic if there is a bijective function $f: V_a\Rightarrow V_b$ such that $(v_i, v_j)\in E_a$ if and only if $(f(v_i), f(v_j))\in E_b$ and $L_a(v_i)$ = $L_b(f(v_i)), L_a(v_j)=L_b(f(v_j))$. 
\end{definition}

\begin{definition}[Static Subgraph Matching]
The static subgraph matching problem is finding all the subgraphs in a data graph $G$ that are isomorphic to a query pattern $Q$. 
\end{definition}

\begin{definition}[Dynamic Graph]
A dynamic graph $G$ = $(G_0, \Delta G)$ is a sequence of edge updates $\Delta G$ = [$\Delta e_0$, $\ldots$, $\Delta e_k$, $\ldots $] applied to an initial graph $G_0$. Each update $\Delta e_k$ = $(v_i, v_j, \oplus)$ inserts or deletes edge $(v_i, v_j)$ from the graph $G_{k}$. The symbol $\oplus$ can be $+$ or $-$, meaning an edge insertion or deletion. Each update $\Delta e_k$ arrives at a random time $T(\Delta e_k)$ $>$ $0$, and $T(\Delta e_{k+1})$ $>$ $T(\Delta e_k)$. The edge updates generate a sequence of graph snapshots [$G_0, $$G_1, G_2, G_3, \ldots $] where $G_{k+1} $= $G_{k}$ $\oplus$ $\Delta e_k$.
\end{definition}


\begin{definition}[Continuous Subgraph Matching]
Continuous subgraph matching (CSM) aims to find the incremental subgraphs $\Delta m_k$ that are isomorphic to a query pattern $Q$ in a dynamic graph $(G_0, \Delta G)$ for each update $\Delta e_{k} \in \Delta G$. 
\end{definition}

%\begin{definition}[Incremental Matching]
%We call the matching procedure for each $\Delta e_k \in \Delta G$ {\em incremental matching}.
%\end{definition}


%-----------------------------------------------------------------------------------------------------------------------------------------%

% \newpage
\iffalse
\subsection{Continuous Subgraph Matching Algorithm}
\label{sec:csk_algo}
Previous systems can be categorized into two types: those that process updates edge by edge and those that process them batch by batch. Readers interested in algorithmic details may refer to~\cite{wei2024gcsm}. Understanding the details of the continuous subgraph matching algorithm is not necessary for following this paper.
\fi
 

%-----------------------------------------------------------------------------------------------------------------------------------------%


\subsection{Existing GPU-based CSM systems}
\label{sec:previous_systems}
%\textcolor{blue}{In this section, we introduce the NVIDIA GPU architecture and CUDA programming model, followed by an overview of prior GPU-based systems.}

A GPU contains tens of streaming multiprocessors (SMs), each with 32-192 CUDA cores depending on the architecture. GPU kernels are organized as grids of thread blocks, where each block contains multiple 32-thread warps. Thread blocks are assigned to SMs for execution, with each SM capable of hosting multiple blocks. Threads within a warp execute in SIMT fashion, and all threads can access the GPU's global memory.

Previous GPU-based CSM systems \cite{wei2024gcsm, qiu2024gpu} are designed to operate with a fixed batch size: they wait until a predefined number of edge updates are accumulated to form a batch, process that batch on the GPU, and then move on to the next one. 
Each batch is processed in three steps:

\begin{itemize}[leftmargin=0.3cm, itemindent=0cm]
\item \textbf{Prepare:} Construct a batch graph $G_b$ from the edges in the batch on the CPU and transfer $G_b$ to the GPU’s global memory. The system now maintains two graphs: the data graph $G_k$ (to be updated) and the batch graph $G_b$.
\item \textbf{Match:} Perform incremental matching for each edge in the batch. Each GPU warp processes one edge and explores its $k$-hop neighborhood in both $G_b$ and $G_k$. Larger batch sizes provide higher parallelism.
\item \textbf{Update:} Merge the edges from $G_b$ into $G_k$, and then sort each updated neighbor list in $G_k$ by vertex ID. Prior GPU systems differ in their data graph placement: GCSM \cite{wei2024gcsm} stores $G_k$ in the CPU’s main memory, whereas GAMMA \cite{qiu2024gpu} stores $G_k$ in the GPU’s global memory.
\end{itemize}
% If a new edge arrives before the previous batch is finished, it waits in the queue. 


% \textcolor{green}{merge the following three paragraphs into one brief intro to GPU architecture..}
% \textcolor{green}{modify/improve the following paragraphs to a single paragraph that explains how previous GPU-based CSM systems work, how are the computations mapped to GPU cores? How are data cached in shared memory? why a batch is needed to achieve good performance? how data are batched on CPU and passed to GPU? etc...}
% The large-scale GPU parallelism occurs in this step.

% These systems adopt algorithms that require the dynamic data graph $G$=$(G_0, \Delta G)$ to keep organized according to certain rules. 
% For example, most prior systems adopt the set operation-based (set intersection $\cap$ and set difference $-$) algorithm, which requires the neighbor lists of $G$ to be sorted by vertex ID. That means inserting a new update will affect the matching results of $\Delta E_t$ if $\Delta E_t$ is still unfinished. 
% The GPU kernels in prior systems \cite{wei2024gcsm, qiu2024gpu} cannot achieve good GPU utilization for a $\Delta E_t$ with a small batch size, they have attempted to exploit intra-$\Delta E_t$ parallelism to saturate the GPU, but none have been effective for small batch sizes. Both CSM and SSM are implemented as a nested loop that traverses a search tree (see section \ref{sec:inter_batch_algo}). There are two intra-$\Delta E_t$ parallel traversal strategies: parallel depth-first search (DFS) and parallel breadth-first search (BFS). DFS is commonly adopted in previous GPU CSM systems \cite{wei2024gcsm, qiu2024gpu}, which requires $|\Delta E_t|$ to be larger than the number of processing units, but that is not a realistic scenario and severely lacks flexibility. BFS or hybrid BFS-DFS \cite{wei2022stmatch, xiang2021cuts} can be used to process a single $\Delta E_t$ with higher parallelism even if $|\Delta E_t|$ is small, but previous studies \cite{wei2022stmatch, xiang2021cuts} have shown that BFS incurs high overhead due to intensive memory writing, huge memory consumption, and threads synchronization, leading to significant performance degradation. 

% Due to the above reasons, it is necessary to enable parallel processing across different $\Delta E_t$. 
% \textcolor{blue}{The graph data structures proposed in prior systems all emphasis on insertion and deletion efficiency but can not hanlde the problem mentioned above. For example, cuStinger ..., Hornet ..., Gunrock ..., Faimgraph ..., GPMA ..., LPMA .... 
% }

% Global memory is accessible by all SMs but has high latency.  Proper use of shared memory and registers is crucial for efficient GPU kernel performance.
% Each CPU has its own main memory which is shared among different GPUs. 
% and registers for thread-private data with the lowest latency.
% Shared memory is exclusive to a block and shared among warps within that block.
% The memory space has a hierarchy of main memory, GPU global memory, and shared memory. 
% Across SMs, active blocks are executed in parallel.
% Within an SM, n warps within a block 32 x n CUDA cores simultaneously, with each warp’s 32 threads executed in parallel by 32 CUDA cores
% CUDA has a thread hierarchy of Kernel $\to$ Block $\to$ Warp $\to$ Thread. 
%NVIDIA GPUs have a hardware hierarchy \cite{cuda} of CPU Host $\to$ GPU $\to$ Streaming Processor (SM) $\to$ CUDA Core. 
%----------------------------------------------------------------------



\section{Overview of DCSM}
\label{sec:overview}


%To addresses the problem discussed in Section~\ref{sec:Challenges}, we propose DCSM. 


Figure~\ref{fig:dcsm_overview} provides an overview of our DCSM system, which consists of three modules: a graph updater (GU), an executor (EX), and a garbage collector (GC). 

Each edge update $\Delta e_{k} = (v_i, v_j, \oplus)$ flows through these three modules, which process multiple updates in a pipelined manner. 
The graph updater (GU) accepts batches of edge updates $\Delta e_{k}$ from the CPU and applies them to the data graph on the GPU using 1-4 warps, where all warps work together to process each batch in parallel.
The executor (EX) performs incremental matching for each $\Delta e_{k}$ and produces the matching results; it runs on thousands of warps and dynamically schedules new updates to idle warps. 
The garbage collector (GC) runs on the GPU and reclaims the memory allocated by GU back to the memory pool.
To ensure correctness, edge updates must preserve their order of arrival—i.e., $\Delta e_{k+1}$ cannot pass through GU, EX, or GC before $\Delta e_{k}$.

Each functional module serves as both a consumer and a producer, and its workload influences the flow speed. 
For instance, if GU is a lightweight module and EX is a heavy module, it will lead to multiple $\Delta e_{k}$ accumulating between GU and EX. Conversely, if GU is heavier than EX, it will cause EX to become idle most time. 
DCSM adopts a warp-specialized design, in which each module is executed by one or more warps on a GPU. 
Each module can also adjust itself based on the accumulation of $\Delta e_{k}$ on its left and right sides. 
For example, if too many $\Delta e_{k}$ accumulate between GU and EX, then GU will pause and wait for the accumulated $\Delta e_{k}$ to be consumed by EX.

% By specifying the number of warps allocated to each module, we can control their consuming and producing speed, thereby enabling effective coordination among different functional modules.

% \begin{itemize}[leftmargin=0.3cm, itemindent=0cm]
% \item \textcolor{green}{point out GU accepts edge updates from CPU, and updates the data graph on GPU, explain how it uses GPU warps...}
% \item \textcolor{green}{explain how it's executed on GPU..}
% \item \textcolor{green}{GU is also executed on GPU? You may want to point out this difference with previous systems.}
% \item \textcolor{green}{I think the drawing of this figure can be improved: use three subfigures and reference them in the text, use annotation?}
% \end{itemize}



\begin{figure}[t]
    \centering
    \includegraphics[scale=0.7, page=6]{./fig/slides-crop.pdf}
    \caption{Overview of DCSM. DCSM consists of three functional modules: Graph Updater (GU), Executor (EX), and Garbage Collector (GC). The edge updates $\Delta e_{k}$ flows through these modules sequentially.}
    \label{fig:dcsm_overview}
\end{figure}



Figure~\ref{fig:dcsm_benefit} illustrates how DCSM processes a sequence of contiguous graph updates.
In this example, the GPU has two parallel computing units: warp~0 and warp~1.
In GCSM~\cite{wei2024gcsm} with a batch size of~2, $\Delta e_{3}$ must wait until $\Delta e_{4}$ arrives to form a batch, which delays the processing of $\Delta e_{3}$.
In contrast, GCSM with a smaller batch size of~1 processes all $\Delta e_{k}$ serially, resulting in lower throughput and longer average response time.
DCSM, by enabling inter-batch parallelism, can flexibly schedule any newly formed batch to an idle warp, thereby improving both response time and throughput. 

The rest of the paper is organized as follows.
Sections~\ref{sec:mvg},~\ref{sec:ex}, and~\ref{sec:gc} describe the three functional modules—GU, EX, and GC, respectively.
Section~\ref{sec:exp} presents the evaluation results. 

% \textcolor{green}{I dont understand this pargraph, try to polish/rewrite it}

% \begin{figure}
%     \centering
%     \includegraphics[scale=0.75, page=5]{./fig/slides-crop.pdf}
%     \caption{\textbf{Top}: GCSM with batch size $1$. \textbf{Middle}: GCSM with batch size $2$. \textbf{Bottom}: DCSM. Eight updates $\Delta e_{1}$ - $\Delta e_{8}$ arrive sequentially. The bar in the figure denotes warp activity along the time axis. Each bar is associated with an update $\Delta e_{i}$, labeled inside the bar. In practice, DCSM uses 1-8 warps for graph updates, while 5000-10000 warps for incremental matching.}
%     \label{fig:dcsm_benefit}
% \end{figure}


\begin{figure}[ht]
    \centering
    \subfloat[GCSM with batch size $1$.]{
        \includegraphics[scale=0.7, page=1]{./fig/compare-crop.pdf}
        \label{fig:gcsm1}
    } \\
    \hspace{2em}
    \subfloat[GCSM with batch size $2$.]{
        \includegraphics[scale=0.7, page=2]{./fig/compare-crop.pdf}
        \label{fig:gcsm2}
    } \\
    \subfloat[DCSM with inter-batch parallelism.]{
        \includegraphics[scale=0.7, page=3]{./fig/compare-crop.pdf}
        \label{fig:dcsm}
    } \hfil
    \caption{Comparison between GCSM and DCSM. Eight updates $\Delta e_{1}$ - $\Delta e_{8}$ arrive sequentially. The bar in the figure denotes warp activity along the time axis. Each bar is associated with an update $\Delta e_{k}$, labeled inside the bar. In practice, DCSM uses 5,000-10,000 warps for incremental matching, while only 1-4 warps are assigned for graph updates.}
    \label{fig:dcsm_benefit}
\end{figure}








%---------------------------------------------------------------------------------------------------------------------------------------------------%
\section{Graph Updater and Multi-Version Graph}
\label{sec:mvg}
% \textcolor{green}{reorg this section to describe the motivation, analysis, and design details of multi-version graph data update..}

\begin{figure*}[h]
    \centering
   \includegraphics[scale=0.53, page=7]{./fig/slides-crop.pdf}
   \caption{The multi-version graph (MVG) data structure and the procedure for updating it with a batch. The MVG before update represents $G_0$ in Figure~\ref{fig:csm}. The vertex ID indexes the $pSlab$, pointing to a $Slab$. Each row of the $Slab$ corresponds to a neighbor array version, storing its creation time ($ct_{min}$), deletion time ($dt_{max}$), and address ($pNb$). Each column of the neighbor array stores the ID, creation time ($ct$), and deletion time ($dt$) of an edge. We use \textbf{$v_x$: ($ct$, $dt$]} to denote a neighbor array version of $v_x$. For example, $v_1$ has two neighbor array versions: $v_1$: ($0$, $3$] and $v_1$: ($3$, $m$] where $m$ means infinity.}
   \label{fig:adjlist}
\end{figure*}

% This section first explains why previous systems~\cite{wei2024gcsm, qiu2024gpu} cannot achieve inter-batch parallelism, and then describes how our system enables it through a multi-version graph data structure. 


All existing CSM systems \cite{wei2024gcsm, qiu2024gpu} on GPUs are derived from static subgraph matching frameworks \cite{wei2022stmatch, chen2022efficient}, and they fail to address data race problems that arise when processing multiple batches concurrently. As a result, these systems are forced to handle $n$ batches ($B_1$, $\ldots$, $B_n$) in a strictly sequential manner: $P(B_1) \rightarrow M(B_1) \rightarrow U(B_1) \rightarrow \ldots \rightarrow P(B_n) \rightarrow M(B_n) \rightarrow U(B_n)$, where $P$, $M$, and $U$ represent the Prepare, Match, and Update phases in Section~\ref{sec:previous_systems}. This serialization stems from the sort operation in the Update phase: executing $U(B_k)$ concurrently with $M(B_k)$ would introduce data races, and $M(B_{k+1})$ must be deferred until $U(B_k)$ finishes, because $M(B_{k+1})$ depends on the data graph state produced by $U(B_k)$.
To overcome this limitation, we introduce a multi-version graph data structure for CSM. 

\noindent
\subsection{Multi-Version Graph (MVG) Data Structure}

Our MVG data structure adopts the classic adjacency list format, but each vertex maintains multiple neighbor arrays of different versions. Figure~\ref{fig:adjlist} shows an example; its caption provides an explanation. This design fully considers the efficiency of the matching algorithm. To satisfy the matching efficiency, each neighbor array is sorted by vertex ID. To enable dynamic extension and shrinkage, slab objects of each vertex are linked together as a list. To avoid the data race problem mentioned above, we maintain multiple neighbor array versions for each vertex. Therefore, inserting a batch of edges in the data graph will not cause data races with the ongoing matching procedures since the newly inserted edges reside in newly created versions of the neighbor arrays. The $ct_{min}$, $dt_{max}$, $ct$, and $dt$ in MVG guide the GPU kernel in determining which version to visit during incremental matching.


\noindent
\subsection{Update Procedure}


Figure~\ref{fig:adjlist} visualizes the graph update process. For three edge updates arriving at times $3$, $5$, and $7$, we group them into a batch. The graph update consists of two steps. (1) Form a batch graph of adjacency list format using the edges in the batch, its neighbor arrays are sorted by edge arrival time, and record the edge update type (insertion or deletion). (2) Update the MVG. Each neighbor array in the batch graph is partitioned into two parts at the first insertion edge. The left part includes the early-arrived edges to be deleted in-place in MVG, while the right part includes the later-arrived edges to be inserted or deleted, but needs to create a new neighbor array version for the MVG. Take the example in Figure~\ref{fig:adjlist}. To update $v_5$'s neighbor array in the MVG, we first update the deletion time of $v_3$ to $5$, then create a new neighbor array version and insert $v_1$ into it.

The Graph Updater (GU) can be configured with two parameters: a waiting time $t$ and a batch size $b$.  Within the time window $t$, GU waits for $b$ edge updates $\Delta e_k$ to arrive to form a batch. If the waiting time $t$ expires before $b$ edge updates arrive, all arrived $\Delta e_k$ form a batch smaller than $b$. A smaller $b$ can have better response time, but if $b$ is too small it will slow down the GU module. Typically, we set $b$ to $64$ and $t$ to a small value (e.g., $0.125\text{ms}$) that does not affect user experience.

% This design considers the complexity of both the algorithm and update procedure. The time complexity of inserting an edge $(v_i, v_j)$ to the MVG data structure is just $O(2 * n)$ where $n$ is the size of neighbor array, while incremental matching has NP-hard time complexity $O(n^{|V(Q)-2|})$. Therefore, GU consumes and produces $\Delta e_k$ much faster than EX in most cases, even though EX is executed in parallel by multiple warps. 

\noindent
\subsection{Parallel Graph Updater}

\label{sec:async_gu}
\begin{figure}[h]
    \centering
   \includegraphics[scale=0.6, page=9]{./fig/slides-crop.pdf}
   \caption{The process of copying three different-sized arrays in parallel using $8$ threads of a warp. $*v_i$ represents the address of $v_i$'s latest neighbor array version in the MVG.}
   \label{fig:copy}
\end{figure}

To prevent the graph updater (GU) from becoming a performance bottleneck for lightweight query workloads, we need to accelerate it on GPU. However, because GPU warps follow the SIMT execution model and the average neighbor array is small, it is difficult to keep all 32 threads in a warp busy. 
More specifically, to merge a batch graph $G_b$ into the MVG $G_k$, for each vertex $v$ in $G_b$, we first copy both $v$'s latest neighbor array from $G_k$ and $v$'s neighbors from $G_b$ into a newly allocated array, which becomes $v$'s updated neighbor version, and then sort this array while excluding deleted vertices. 
A naive approach assigns one warp for the task, with each thread copying one array element. However, since the average neighbor size in most real-world graphs is only 2 to 3, only a few threads are active during the execution. To improve thread utilization, we would like the 32 threads in a warp to copy multiple neighbor arrays in parallel. 


To copy multiple arrays in parallel with one warp, we use the following strategy.
Figure~\ref{fig:copy} shows how a warp copies three arrays in parallel; specifically, it copies $v_1$: ($0$, $3$], $v_3$: ($0$, $3$] and $v_5$: ($0$, $7$] for the update procedure shown in Figure~\ref{fig:adjlist}. 
In this example, \texttt{np}, \texttt{size}, etc., are register variables, and cross-thread operations on them use CUDA warp-level primitives.
Initially, each thread stores the address (\texttt{np}) and size (\texttt{size}) of a distinct array. 
We first compute the prefix sum (\texttt{ps}) over the \texttt{size} values. 
Then, we broadcast \texttt{ps} and \texttt{np} into \texttt{bps} and \texttt{bnp} based on \texttt{size} value. 
For example, since thread $2$ has \texttt{size} = 3, $*v_3$ is broadcast to three copies in \texttt{bnp}.
Next, we compute the \texttt{idx} as \texttt{idx} = $tid$ - \texttt{bps}, where $tid$ is thread id. 
Using \texttt{idx} and \texttt{bnp}, each thread maps to a unique element within its assigned array. 
For instance, thread $3$ accesses the neighbor at index $2$ in $v_3$'s array, which is vertex $v_5$.
If the sum of neighbor sizes exceeds the number of threads within a warp, the warp iterates to complete the copy. 
In our actual system design, we treat $n$ warps as a large "super-warp" with $n \times 32$ threads, copying multiple neighbor arrays in parallel. 


We also process two consecutive batches in a pipelined manner. Specifically, data copying is implemented using the \texttt{memcpy\_async} instruction, allowing it to overlap with the sorting phase of the previous batch.




% \textcolor{blue}{Second, we enable parallel data copying to fully exploit memory bandwidth. We use 32n threads from n warps to update multiple neighbor arrays from the batch graph to the MVG in parallel. For special neighbor arrays with larger sizes ($>$ 32), we assign a dedicated warp to process each one. We use CUDA vectorized memory access to perform data loading and writing in 128-bit transactions. This not only improves the efficiency of GU but also allows us to control the processing speed of GU by allocating different numbers of warps.}

% These two optimizations are beneficial for cases where GU becomes the bottleneck instead of EX, especially when the workload of EX is not extremely heavy.



% \textcolor{blue}{First, we enable the graph updater to consume $\Delta e_k$ asynchronously. This design enables time overlap between the creation of new neighbor array versions and the matching procedures that use them, thereby hiding memory access overhead behind the matching process. As a result, some $\Delta e_k$ between GU and EX, or those currently being processed by EX, may not have their updated neighbor arrays ready. A warp in EX will stall and wait if it accesses a neighbor array version that has not yet been written to memory.}



% For example, for a batch $B_k$, $M(B_k)$ depends on the batch graph formed by $F(B_k)$. $U(B_k)$ can only update the neighbor arrays after $M(B_k)$ finishes because the insertion and sorting operations in $U(B_k)$ would cause data races with $M(B_k)$. Furthermore, $M(B_{k+1})$ must wait for $U(B_k)$ to complete, because $M(B_{k+1})$ is based on the data graph version after $U(B_k)$.

% \subsection{Problem Analysis}
% \label{sec:problem_analysis}

% \textcolor{red}{explain what the problem is...}

% \textcolor{red}{remove this paragraph: For example, for a batch $B_k$, $M(B_k)$ depends on the batch graph formed by $F(B_k)$. $U(B_k)$ can only update the neighbor arrays after $M(B_k)$ finishes because the insertion and sorting operations in $U(B_k)$ would cause data races with $M(B_k)$. Furthermore, $M(B_{k+1})$ must wait for $U(B_k)$ to complete, because $M(B_{k+1})$ is based on the data graph version after $U(B_k)$.}

% \textcolor{red}{remove this: To enable inter-batch parallelism, we need to remove the sorting operation in $U(B_k)$. We can replace the WCOJ algorithm used in previous systems with
% an edge-extension algorithm, which doesn't require the neighbor arrays to be sorted. However, edge-extension generates significantly more intermediate results than WCOJ, slowing down the matching step. Alternatively, we could store the neighbor arrays as hash sets, but previous studies have shown that hash-set-based subgraph matching systems cannot achieve optimal performance on GPU platforms. We evaluate both approaches as baselines in the experimental section. Another potential approach is to exploit high parallelism within a small batch, such as traversing the search space in BFS order. However, previous studies \cite{wei2022stmatch} have shown that BFS incurs high overhead due to intensive memory writes, large memory consumption, and thread synchronization, leading to significant performance degradation. Therefore, we need an entirely new solution.}




\section{Executor} 
\label{sec:ex}
% \begingroup
% \color{blue}

The Executor (EX) consumes updates $\Delta e_k$ produced by GU and performs incremental matching for each $\Delta e_k$. Since subgraph matching is NP-hard \cite{ullmann1976algorithm}, the EX is the performance bottleneck of the system and needs to be parallelized across multiple GPU warps. 
This section explains how we adapt an existing subgraph matching kernel \cite{wei2022stmatch, chen2022efficient} for EX to ensure both correctness and efficiency in CSM.

%However, our system faces two challanges that did not exist in previous systems: (1) Our matching program is based on a novel multi-version graph (MVG) that we propose, rather than the classic graph data structure used in previous systems. Thus, we must ensure the correctness and efficiency of our program on the MVG. (2) The updates $\Delta e_k$ before EX increase dynamically as the GU produces. Therefore, we need to dynamically schedule newly added $\Delta e_k$ to idle warps.

%In the following three subsections, we first prove the correctness of our matching algorithm on the MVG, then demonstrate its efficiency, and finally illustrate how we dynamically schedule newly arriving $\Delta e_k$ to idle warps.

\noindent
\subsection{Correctness Guarantee}
\label{sec:mvnv}

For any update $\Delta e_k = (v_{i}, v_{j}, \oplus)$ that arrives at time $T(\Delta e_k)$, its incremental matching will access the neighbor arrays of $v_i$ and $v_j$'s k-hop neighbor vertices. However, our multi-version graph (MVG) data structure introduces a problem: which neighbor array version should be used when visited? Property~\ref{lemma1} to Property~\ref{lemma3} provide guidance on how the Executor's GPU kernels can correctly access the MVG. 

% \textcolor{red}{this section doesn't look good. Proofs seem to be overuse. They are not appropriate for explanating system properties. Replace them with plain text and (even better) figures to explain the properties.  }


\begin{property}
\label{lemma1}
For an edge $e_k$ that is added to the data graph at time $t_1$ and deleted at time $t_2$, the incremental matching for updates arriving within $(t_1, t_2]$ should see $e_k$ as existing in the data graph. Conversely, the incremental matching for updates outside $(t_1, t_2]$ should not.
\end{property}


For a specific edge $e_k$ = $(v_{i}, v_{j})$, there is exactly one edge insertion $\Delta e_k^+ = (v_{i}, v_{j}, +)$ and at most one edge deletion $\Delta e_k^- = (v_{i}, v_{j}, -)$. We have $T(\Delta e_k^+) = 0$ if $e_k$ exists in the initial data graph $G_0$, and $T(\Delta e_k^-) = \infty$ (or $m$) if $e_k$ is never deleted. For both $\Delta e_k^+$ and $\Delta e_k^-$, the \textbf{Update} step starts after \textbf{Matching} (see Section~\ref{sec:previous_systems}), and we regard each edge update as a single batch. Therefore, $e_k$ is invisible to $M(\Delta e_k^+)$, and $e_k$ is visible to $M(\Delta e_k^-)$. In addition, it is obvious that $e_k$ is visible to $M(\Delta e_x)$ for all $\Delta e_x$ such that $T(\Delta e_k^+) < T(\Delta e_x) < T(\Delta e_k^-)$. Even though our multi-version graph allows \textbf{Update} and \textbf{Matching} to execute in parallel, the visibility order mentioned above remains unchanged. In summary, the incremental matching for updates arriving within $(T(\Delta e_k^+), T(\Delta e_k^-)]$ should see $e_k$ as existing in the data graph.

\begin{figure}[h]
    \centering
   \includegraphics[scale=0.8, page=16]{./fig/slides-crop.pdf}
   \caption{Visualization of edge $e_k$'s lifetime and visibility across other updates.}
   \label{fig:lifetime}
\end{figure}


We therefore define the lifetime of an edge and a neighbor array version. The \textbf{lifetime of an edge $e_k$} is $(t_1$, $t_2]$, where $e_k$ is added to the data graph at time $t_1$ and deleted at time $t_2$. \textbf{The lifetime of a neighbor array version} in a multi-version graph is $(ct_{min}, dt_{max}]$, where $ct_{min}$ and $dt_{max}$ are the minimum creation time and maximum deletion time of entries in this neighbor array, respectively. Figure~\ref{fig:lifetime} visualizes the lifetime of an edge $e_k$ and which edge updates can see $e_k$ existing in the data graph.

\begin{property}
\label{lemma2}
To ensure correctness, the incremental matching for an update $\Delta e_k$ should only visit edges whose lifetime $(t_1, t_2]$ contains $T(\Delta e_k)$ ($T(\Delta e_k)$ $\in$ $(t_1, t_2]$). 
\end{property}


According to Property~\ref{lemma1}, an edge $e_x$ in the data graph should only be visible to the incremental matching procedure for updates arriving within $e_x$'s lifetime $(t_1, t_2]$. In other words, for an update $\Delta e_k$ arriving at time $T(\Delta e_k)$, its matching procedure should only visit edges whose lifetime $(t_1, t_2]$ contains $T(\Delta e_k)$.


\begin{property}
\label{lemma3}
To ensure correctness, the incremental matching for an update $\Delta e_k$ should only visit the neighbor array versions whose lifetime $(ct_{min}$, $dt_{max}]$ contains $T(\Delta e_k)$ and filter out the edges whose lifetime does not contain $T(\Delta e_k)$. 
\end{property}


In the MVG shown in Figure~\ref{fig:adjlist}, each $(ct, dt]$ pair represents a segment of an edge's lifetime. For example, the lifetime of edge $(v_1, v_0)$ consists of two segments, $(0, 3]$ and $(3, m]$, while edge $(v_1, v_5)$ has a lifetime of $(7, m]$. The lifetime of a neighbor array covers the time ranges of edges in it. Therefore, according to Property~\ref{lemma2}, the valid neighbor edges must be in neighbor array versions whose lifetime $(ct_{min}, dt_{max}]$ contains $T(\Delta e_k)$.


\begin{example}
For an update arriving at time $6$, if its matching procedure visits $v_5$'s neighbor array, it should visit the neighbor array version with lifetime $(0, 7]$ since $6 \in (0, 7]$. However, the neighbor vertex $v_3$ in this version should be ignored because its deletion time is $5$, which means $v_3$ does not exist at time $6$.
\end{example}

\noindent
\subsection{Continuous Subgraph Matching-Specific Optimizations}

\subsubsection{Efficient Neighbor Array Access}
The slab design in the multi-version graph (MVG), together with CUDA warp-level primitives, enables our system to efficiently access neighbor arrays. Each slab in the MVG contains 32 entries, and multiple slabs are connected as a linked list. 
The matching for an update $\Delta e_k$ is performed by a single warp. 
When a warp accesses a vertex's neighbor array, it traverses the slab linked list.
For each slab, the warp's 32 thread lanes check in parallel whether $T(\Delta e_k)$ falls within the lifetime of any of the 32 slab entries. We leverage the warp-level primitives \texttt{\_\_ballot\_sync} and \texttt{\_\_ffs} to perform parallel checks. 
The two functional modules, GU and GC, collaboratively make the number of valid entries in a slab list dynamically grow and shrink. For almost all graphs, the average number of valid entries per slab list remains between 1 and 5, which is much smaller than 32. Therefore, compared with previous systems, our design does not increase the time complexity of neighbor array access.


\subsubsection{Dynamic Task Scheduler}
The updates $\Delta e_k$ before EX increase dynamically as the GU produces. Therefore, we need to dynamically schedule newly added $\Delta e_k$ to idle warps. To solve this problem, we propose a dynamic scheduling strategy. For a $\Delta e_k$ to be consumed by EX, $n$ idle warps will simultaneously compete for this $\Delta e_k$. The warp wins will asynchronously execute this $\Delta e_k$, then the remaining $ n-1$ idle warps will compete for the next $\Delta e_k$. Once a warp finishes the matching of a $\Delta e_k$, this warp returns to the competition status to compete for its next $\Delta e_k$. Therefore, multiple $\Delta e_k$ are dynamically scheduled to the idle warps, which can ensure the load balance among warps. 

\subsubsection{Output of Executor}
EX not only forwards each $\Delta e_k$ it processes to GC, but also outputs the matching result of each $\Delta e_k$. The matching results of an edge insertion mean that we obtain additional valid matches in the data graph, while the matching results of an edge deletion mean that previously valid matches are no longer valid in the data graph. In some scenarios, it is necessary to obtain the matching result for an entire batch. We can compute the union of the matching results of all individual updates in the batch to obtain the overall batch matching result. 





\section{Garbage Collector} 
\label{sec:gc}


The multiple neighbor array versions allocated by the graph updater (GU) consume a large amount of GPU memory. 
These neighbor arrays should be released once they are no longer in use; otherwise, GPU memory usage will keep growing. 
This section describes our strategy for releasing these neighbor array versions. 

\subsection{Release Conditions}
\label{sec:rc}

A neighbor array version with lifetime $(ct_{min}$, $dt_{max}]$ can be released once all updates $\Delta e_k$ whose $T(\Delta e_k)$ $\in$ $(ct_{min}$, $dt_{max}]$ have finished their matching tasks.

We can see this point from another perspective of Property~\ref{lemma3}. According to Property~\ref{lemma3}, a neighbor array will be accessed by $\Delta e_k$'s matching procedure if $T(\Delta e_k)$ lies within the array's lifetime $(ct_{min}$, $dt_{max}]$. From another viewpoint, once all updates $\Delta e_k$ with $T(\Delta e_k) \in (ct_{min}$, $dt_{max}]$ have finished their matching tasks, this neighbor array version will no longer be used and can be deleted.


\subsection{Garbage Collection Graph (GCG)}

The garbage collector (GC) operates on a garbage collection graph (GCG), which helps GC release neighbor array versions that are no longer in use.

\begin{figure}[h]
    \centering
        \includegraphics[scale=0.63, page=10]{./fig/slides-crop.pdf}
   \caption{A garbage collection graph (GCG) built from the 3 updates $\Delta e_0$, $\Delta e_1$, $\Delta e_2$ in Figure~\ref{fig:adjlist}.}
   \label{fig:aux}
\end{figure}

The GCG's vertices include a \textit{Zero} vertex and all $\Delta e_k$ that have been processed by the graph updater (GU).  
Each vertex in GCG is an edge update $\Delta e_k$ timestamped with its arrival time, and the \textit{Zero} vertex is a dummy edge update timestamped with $0$. 
Each edge in GCG corresponds to a "deletable" neighbor array. Once we create a new neighbor version on MVG, the previous neighbor version becomes "deletable". 
For each deletable neighbor array $v_x$:($ct_{min}$, $dt_{max}$], we add a directed edge to GCG from the vertex (edge update) whose creation time is $ct_{min}$ to the vertex whose creation time is $dt_{max}$.



Figure~\ref{fig:aux} shows a GCG example; it has four vertices, three of which are the edge updates in Figure~\ref{fig:adjlist}. 
Since GU has created $v_1$:$(3, m]$, $v_3$:$(3, m]$, and $v_5$:$(7, m]$, the arrays $v_1$:$(0, 3]$, $v_3$:$(0, 3]$, and $v_5$:$(0, 7]$ become deletable. Therefore, three edges are added to the GCG and are marked with their corresponding deletable neighbor arrays.


The GCG guides the release of neighbor arrays. 
An edge $v_x$:($ct_{min}$, $dt_{max}$] in the GCG corresponds to a neighbor array version, where the vertices (edge updates) below this edge represent updates $\Delta e_k$ with $T(\Delta e_k) \in (ct_{min}, dt_{max}]$.
According to Section~\ref{sec:rc}, once all edge updates below a GCG edge have completed incremental matching, the corresponding neighbor array can be released.
For example, in Figure~\ref{fig:aux}, if $\Delta e_0$, $\Delta e_1$, and $\Delta e_2$ under $v_5$:(0, 7] have finished their incremental matching, the neighbor array corresponding to $v_5$:(0, 7] can be released.


\subsection{Implementation}

In this section, we introduce how the garbage collector (GC) runs on the garbage collection graph (GCG), and its logic to release neighbor arrays.

\begin{figure}[t]
    \centering
    \subfloat[The incremental matching of $\Delta e_0$ is finished, $v_1$:$(0, 3 \rbrack $ and $v_3$:$(0, 3 \rbrack $ can be released.]{
        \includegraphics[scale=0.63, page=11]{./fig/slides-crop.pdf}
        \label{fig:gc_1}
    } \hfil
    \subfloat[The incremental matching of $\Delta e_2$ is finished.]{
        \includegraphics[scale=0.63, page=12]{./fig/slides-crop.pdf}
        \label{fig:gc_2}
    } \\
    \subfloat[The incremental matching of $\Delta e_1$ is finished, $v_5$:$(0, 7 \rbrack $ can be released.]{
        \includegraphics[scale=0.63, page=13]{./fig/slides-crop.pdf}
        \label{fig:gc_3}
    } \hfil
    \subfloat[A new $\Delta e_3$ = $(v_1, v_4, +)$ is appended to the end of GCG.]{
        \includegraphics[scale=0.63, page=14]{./fig/slides-crop.pdf}
        \label{fig:gc_4}
    }
    \caption{(a)–(d) illustrate how GC releases neighbor arrays on the GCG in Figure~\ref{fig:aux}. This GCG is stored in an adjacency-list format.}
    \label{fig:gc_step}
    %\vspace{-1em}
\end{figure}

GCG is stored in adjacency list format and manipulated collaboratively by GU, EX, and GC. 
GU appends vertices to GCG for each incoming $\Delta e_k$ and appends edges for each deletable neighbor array created.
EX marks whether each edge update in GCG has finished its incremental matching, and GC releases the corresponding neighbor array versions based on these marks.
GC maintains a GC pointer pointing to $\Delta e_k$ such that all incremental matchings from $\Delta e_0$ through $\Delta e_k$ have been completed. 
Once the GC pointer reaches a $\Delta e_k$, the neighbor array versions associated with the edges starting from $\Delta e_k$ will be released.

Figure~\ref{fig:gc_step} shows how GC releases the neighbor arrays on the GCG in Figure~\ref{fig:aux}. 
In (a), once the incremental matching of $\Delta e_0$ is completed, we can move the GC pointer to $\Delta e_0$, and then $v_1$:$(0, 3]$ and $v_3$:$(0, 3]$ are released by GC.
In (b), the incremental matching of $\Delta e_2$ is finished, but the GC pointer cannot be moved to it since $\Delta e_1$ before $\Delta e_2$ is unfinished.
In (c), once the incremental matching of $\Delta e_1$ is finished, the GC pointer is moved to $\Delta e_2$, and thus $v_5:(0, 7]$ is released by GC.
In (d), an edge $\Delta e_3$ = $(v_1, v_4, +)$ at time $T(\Delta e_3)$ = $9$ is added to the GCG by GU. This edge insertion creates new neighbor versions for both $v_1$ and $v_4$. Therefore, $\Delta e_3$ has two out-edges, each corresponding to a newly created deletable neighbor array.


As described above, the condition for the GC pointer to move to $\Delta e_k$ is that the incremental matchings of $\Delta e_0$ to $\Delta e_{k-1}$ are all completed, which means that the neighbor arrays corresponding to out-edges of $\Delta e_k$ can be released. This logic is also based on the release condition in Section~\ref{sec:rc}. 

Since edge updates are added dynamically, GCG is a dynamic graph. 
Each vertex in this dynamic graph has at most two out-edges, so the overhead of maintaining this dynamic graph is very light. Unlike GU, which involves data copying, and EX, which performs NP-hard space search, GC is a lightweight module that does not pose any performance issues.




% Each vertex ($\Delta e_k$) in GCG has two statuses: finished or ongoing, which means the incremental matching on it has been finished or unfinished. 
% In Fig.~\ref{fig:aux}, $\Delta e_0$, $\Delta e_2$ is finished while $\Delta e_1$ is still ongoing. å
% the range of the edge $v_3:(0, 3]$ is $(0, 3]$, and there are two only one vertices $\Delta e_0$ in this range and $\Delta e_0$ is finished, so the neighbor array $v_3:(0, 3]$ in MVG can be deleted. Similarly, $v_1:(0, 3]$ in MVG can also be deleted. åå
% The neighbor array $v_5:(0, 7]$ in MVG can not be deleted since the matching of $\Delta e_1$ is still ongoing. The garbage collector traverses the GCG to release the neighbor arrays based on the logic described above. A vertex marked as finished will be consumed and discarded by the GC, and any edge pointing to it will be redirected to point to the \textit{Zero} vertex.

% 

\section{Experimental Results}
\label{sec:exp}

In this section, we first introduce the experimental setup, then compare the performance of DCSM with previous systems in terms of throughput and response time, and finally evaluate the effectiveness of our proposed optimizations through ablation studies.
%______________________________________________________________________________________________________________________________________________________%
\subsection{Experimental Setup}
\label{sec:exp_setup}

% Table generated by Excel2LaTeX from sheet 'Graph'
\begin{table}[htbp]
    \centering
    \ssmall
    \caption{Graph datasets.}
      \begin{tabular}{c|c|c|c|c}
      \hline
      \textbf{Graphs} & \textbf{Abbr.} & \textbf{\# nodes} & \textbf{\# edges} & \textbf{Max deg.} \bigstrut\\
      \hline\hline
      \textbf{Enron} & en    & 37K   & 183K  & 1383 \bigstrut[t]\\
      \textbf{Amazon} & az    & 334K  & 0.9M  & 549 \\
      \textbf{DBLP} & db    & 317K  & 1.0M  & 343 \\
      \textbf{Netflow} & nf    & 3.1M  & 2.9M  & 0.2M \\
      \textbf{LSBench} & lb    & 5.2M  & 20.3M & 2.3M \\
      \textbf{LiveJournal} & lj    & 4.0M  & 34.7M & 14815 \bigstrut[b]\\
      \hline
      \end{tabular}%
    \label{tab:datasets}%
\end{table}%

  
\noindent
\textbf{Platform:} All experiments are finished on a host with two Intel Xeon Gold 6226R 2.9GHz CPUs (32 cores in total) and Nvidia RTX3090 GPU. The CPUs have 512GB RAM. Each GPU has 24GB global memory. All experiments run on Ubuntu-20.04. DCSM was developed in C++ and CUDA. The CPU code was compiled using GCC 9.4.0 with O3 optimization, and the GPU code was compiled using NVCC 11.6. 

\noindent
\textbf{Datasets and query patterns:} 
Table~\ref{tab:datasets} lists the data graphs used in our experiments. All the data graphs are real-world graphs. Enron, Amazon, DBLP, and LiveJournal are networks of ground-truth communities from the SNAP dataset \cite{snapnets}. Netflow \cite{caida2019caida} is a graph of passive traffic traces. LSBench \cite{le2012linked} is a synthesized graph social network from multiple sources. All graphs are simplified to simple undirected graphs. The dynamic graphs are generated by selecting 45\% of the edges as insertion edges and 5\% as deletion edges. The nine query patterns from size-$3$ to size-$5$ used in experiments are shown in Figure~\ref{fig:queries}. The experiments include labeled and unlabeled matching. For labeled matching, we randomly assign five labels to the data graph and query pattern.

\begin{figure}[h]
    \centering
     %   \vspace{-.5em}
    % \hspace{-1em}
   \includegraphics[scale=0.42, page=15]{./fig/slides-crop.pdf}
   \caption{Query patterns for evaluation.}
   \label{fig:queries}
\end{figure}

\noindent
\textbf{Baselines:} 
We select three recent work, RapidFlow \cite{sun2022rapidflow}, CaLiG \cite{yang2023fast}, and NewSP \cite{li2024newsp} as our CPU baselines. They are three  state-of-the-art open-sourced continuous subgraph matching systems on the CPU, and have reported better performance than other CPU systems, including Graphflow \cite{kankanamge2017graphflow}, TurboFlux \cite{kim2018turboflux}, SJ-Tree \cite{choudhury2015selectivity}, IEDyn \cite{idris2017dynamic}, and Symbi \cite{min2021symmetric}. Each of them features different optimizations. 

There are two previous GPU systems: GCSM \cite{wei2024gcsm} and GAMMA \cite{qiu2024gpu}. We summarize their performance and differences in Table~\ref{tab:baseline}. Since GAMMA is not totally open-sourced, we reimplement GAMMA as GAMMAr. We evaluate them following the experimental settings in the GAMMA paper. GCSM outperforms GAMMA due to two additional optimizations absent in GAMMA: code generation, which simplifies the program, and dynamic scheduling, which reduces load imbalance among thread blocks. GCSM consistently outperforms others in all test cases, so we use GCSM as our GPU baseline. 

We also evaluate two straightforward but suboptimal strategies that can enable inter-batch parallelism: Edge-Extension and Hash-Indexing. 
Edge-Extension extends subgraphs by one edge at a time, so it does not require the neighbor arrays to be sorted, but it generates significantly more intermediate results than our algorithm, slowing down the matching step. 
Hash-Indexing stores the neighbor arrays as hash sets, but previous studies \cite{chen2022efficient} have shown that it cannot achieve optimal performance on GPU platforms.
We evaluate both approaches as baselines in the experimental section.

Another potential baseline approach is to exploit high parallelism within a small batch, such as traversing the search space in BFS order. However, previous studies \cite{wei2022stmatch, xiang2021cuts} have already extensively evaluated BFS and shown that it incurs high overhead due to intensive memory writes, large memory consumption, and thread synchronization, leading to significant performance degradation. Therefore, we did not include BFS in our baseline.

% Table generated by Excel2LaTeX from sheet 'Sheet3'
\begin{table}[htbp]
  \centering
  \ssmall
  \setlength{\tabcolsep}{2pt}
  \caption{Comparison of baselines. \textit{Avg Time} is the average execution time on four data graphs and query patterns used in GAMMA. The \textit{Avg Time} for GAMMA is taken from the original paper, while the \textit{Avg Time} of others is obtained from our own experiments.}
    \begin{tabular}{c|c|c|c|c|c|c}
    \toprule
          & \multicolumn{1}{p{2.75em}|}{\textbf{Avg \newline{}Time}} & \multicolumn{1}{p{5em}|}{\textbf{Work\newline{}Stealing}} & \multicolumn{1}{p{5em}|}{\textbf{Dynamic\newline{}Scheduling}} & \multicolumn{1}{p{5em}|}{\textbf{Warp \newline{}Centric}} & \multicolumn{1}{p{5em}|}{\textbf{Redundancy \newline{}Elimination}} & \multicolumn{1}{p{5em}}{\textbf{Code\newline{}Generation}} \\
    \midrule
    \textbf{GAMMA} & 0.82  & $\checkmark$     & x     & $\checkmark$     & $\checkmark$     & x   \bigstrut[b]\\
    \textbf{GAMMAr} & 0.97  & $\checkmark$     & x     & $\checkmark$     & $\checkmark$     & x   \bigstrut[b]\\
    \textbf{GCSM} & 0.53  & $\checkmark$     & $\checkmark$     & $\checkmark$     & $\checkmark$     & $\checkmark$   \bigstrut[b]\\
    \bottomrule
    \end{tabular}%
  \label{tab:baseline}%
\end{table}%

\noindent
\textbf{Settings:}
Each GPU launches 82 thread blocks, each containing 2048 threads, which fully utilize the available active threads.
RapidFlow, CaLiG, and NewSP run on a single thread because their algorithm can only process edge updates one by one, and their implementation is based on a single-threaded design. All systems adopt the same matching order of the query pattern. Unless otherwise specified, we set the batch size each time consumed by the DCSM Graph Updater (GU) to $64$, and set its timeout threshold $x$ to 0.125ms. We allocate 4 warps to the Graph Updater, 1 warp to the Garbage Collector, and all remaining warps to the Executor. GCSM uses the same 82 blocks as DCSM, each containing 2048 threads, but all warps are used for matching.


%______________________________________________________________________________________________________________________________________________________%

\subsection{Throughput}
\begin{table}[htbp]
  \centering
  \ssmall
  \setlength{\tabcolsep}{3.5pt}
  \caption{Execution time of different systems for matching unlabeled and labeled query patterns. The '-' indicates a timeout after 8 hours. The default time unit for unlabeled matching is seconds, while for labeled matching it is milliseconds. The 's' after a number denotes seconds}
    \begin{tabular}{c|c|c|c|c|c|c|c|c|c|c|c}
    \hline
          & \textit{\textbf{G}} & \textbf{System}  & \textbf{Q1} & \textbf{Q2} & \textbf{Q3} & \textbf{Q4} & \textbf{Q5} & \textbf{Q6} & \textbf{Q7} & \textbf{Q8} & \textbf{Q9} \bigstrut\\
    \hline
    \hline
    \multirow{24}[8]{*}{\begin{sideways}\textbf{Unlabeled}\end{sideways}} & \multirow{6}[2]{*}{\textit{\textbf{en}}} 
                  & \textbf{Rapidflow} & 0.16 & 0.28 & 4.6 & 9.1 & 3.2 & 2.7 & 1.7 & 82.3 & 3.8 \bigstrut[t]\\
          &       & \textbf{CaliG} & 2.4 & 3.1 & 146 & 7.2 & 162 & 394 & 53 & 24332 & 1116 \\
          &       & \textbf{NewSP} & 5.2   & 1.1   & 864   & 13    & 110   & 41    & 27    & -     & 471 \\
          &       & \textbf{HashIdx}  & 0.17 & 0.2   & 3.3   & 3.8   & 1.8   & 3.4   & 0.5.  & 9.9   & 2.7  \\
          &       & \textbf{Edge-Ext}   & 0.1  & 0.1   & 1.5   & 2.1   & 1.6   & 8.4   & 8.8   & 92    & 12 \\
          &       & \textbf{DCSM}       & 0.1  & 0.1   & 1.5   & 2.1   & 0.7   & 1.2   & 0.2   & 7.1   & 1.2 \bigstrut[b]\\
\cline{2-12}          & \multirow{6}[2]{*}{\textit{\textbf{az}}} 
                  & \textbf{Rapidflow} & 1.1   & 2.4   & 3.7   & 1.9   & 4.6   & 5.2   & 4.2   & 7.8   & 5.1 \bigstrut[t]\\
          &       & \textbf{CaliG} & 4.1   & 5.7   & 15    & 7.8   & 32    & 39    & 33    & 230   & 379 \\
          &       & \textbf{NewSP} & 2.2   & 1.9   & 28    & 5.2   & 13    & 5.1   & 4.6   & 92    & 8.4 \\
          &       & \textbf{HashIdx}  & 0.25 & 0.27 & 0.36 & 0.23 & 0.31 & 0.58 & 0.22 & 0.39 & 0.71 \\
          &       & \textbf{Edge-Ext} & 0.09  & 0.1   & 0.1   & 0.1   & 0.1   & 0.5   & 1.2   & 5.5   & 3.2 \\
          &       & \textbf{DCSM} & 0.09  & 0.1   & 0.2   & 0.1   & 0.2   & 0.2   & 0.1   & 0.3   & 0.4 \bigstrut[b]\\
\cline{2-12}          & \multirow{6}[2]{*}{\textit{\textbf{db}}} 
                  & \textbf{Rapidflow} & 1.1   & 2.6   & 5.2   & 3.2   & 5.5   & 6.7   & 6.6   & 29    & 16 \bigstrut[t]\\
          &       & \textbf{CaliG} & 5.4   & 7.8   & 54    & 12    & 102   & 152   & 101   & 5202  & 7638 \\
          &       & \textbf{NewSP} & 4.8   & 3.1   & 181   & 9.1   & 64    & 45    & 76    & 3861  & 7363 \\
          &       & \textbf{HashIdx}  & 0.16 & 0.27 & 1.6 & 0.26 & 0.96 & 1.4 & 1.4 & 4.8 & 1.6  \\
          &       & \textbf{Edge-Ext} & 0.05  & 0.1   & 1.2   & 0.2   & 1.2   & 2.8   & 3.4   & 116   & 11 \\
          &       & \textbf{DCSM} & 0.05  & 0.1   & 1.2   & 0.2   & 0.5   & 0.6   & 0.6   & 1.8   & 0.6 \bigstrut[b]\\
\cline{2-12}          & \multirow{6}[2]{*}{\textit{\textbf{nf}}} 
                  & \textbf{Rapidflow} & 10.1  & 4.4   & 7.2   & 1278  & 5.8   & 5.5   & 5.1   & 6.2   & 5.9 \bigstrut[t]\\
          &       & \textbf{CaliG} & 247   & 58    & 238   & 1508   & 88    & 1678  & 1525  & 244   & 206 \\
          &       & \textbf{NewSP} & 1654  & 2.7   & 1384  & 4027  & 480   & 5.8   & 1.8   & 803   & 2.1 \\
          &       & \textbf{HashIdx}  & 2.7 & 0.93 & 2.5 & 464 & 1.7 & 2.3 & 1.4 & 2.5 & 3.5  \\
          &       & \textbf{Edge-Ext} & 1.2   & 0.6   & 1.2   & 232   & 1.2   & 1.1   & 1.7   & 21    & 4.8 \\
          &       & \textbf{DCSM}     & 1.2   & 0.6   & 1.2   & 232   & 0.9   & 0.9   & 0.8   & 1.7   & 1.2 \bigstrut[b]\\
    \hline
    \hline
    \multirow{36}[12]{*}{\begin{sideways}\textbf{Labeled}\end{sideways}} & \multirow{6}[2]{*}{\textit{\textbf{en}}} 
                  & \textbf{Rapidflow} & 27    & 78    & 37    & 38    & 133   & 102   & 127    & 996   & 42 \bigstrut[t]\\
          &       & \textbf{CaliG} & 32    & 44    & 188   & 72    & 299   & 408   & 164   & 7190  & 555 \\
          &       & \textbf{NewSP} & 95    & 61    & 1283  & 135   & 306   & 125   & 105   & 6633  & 210 \\
          &       & \textbf{HashIdx} & 20 & 9.4 & 19 & 23 & 28 & 49 & 55 & 221 & 80  \\
          &       & \textbf{Edge-Ext} & 7     & 6     & 10    & 11    & 10    & 50    & 62    & 98    & 168 \\
          &       & \textbf{DCSM} & 7     & 6     & 10    & 11    & 10    & 21    & 21    & 91    & 27 \bigstrut[b]\\
\cline{2-12}          & \multirow{6}[2]{*}{\textit{\textbf{az}}} 
                  & \textbf{Rapidflow} & 71    & 110   & 171   & 98    & 174   & 166   & 104   & 204   & 107 \bigstrut[t]\\
          &       & \textbf{CaliG} & 91    & 174   & 141   & 114   & 241   & 334   & 91    & 378   & 387 \\
          &       & \textbf{NewSP} & 283   & 263   & 359   & 309   & 324   & 301   & 274   & 433   & 267 \\
          &       & \textbf{HashIdx} &  91 & 43 & 48 & 71 & 85 & 53 & 88 & 50 & 66 \\
          &       & \textbf{Edge-Ext} & 31    & 32    & 31    & 33    & 33    & 49    & 52    & 178   & 395 \\
          &       & \textbf{DCSM} & 31    & 32    & 31    & 33    & 33    & 32    & 32    & 32    & 45 \bigstrut[b]\\
\cline{2-12}          & \multirow{6}[2]{*}{\textit{\textbf{db}}} 
                  & \textbf{Rapidflow} & 97    & 123   & 172   & 111   & 184   & 238   & 122   & 372   & 179 \bigstrut[t]\\
          &       & \textbf{CaliG} & 114   & 177   & 252   & 172   & 433   & 647   & 997   & 3945  & 13s \\
          &       & \textbf{NewSP} & 299   & 282   & 704   & 352   & 498   & 433   & 494   & 3885  & 4.8s \\
          &       & \textbf{HashIdx} & 53 & 56 & 82 & 35 & 60 & 106 & 53 & 85 & 69  \\
          &       & \textbf{Edge-Ext} & 31    & 30    & 31    & 27    & 30    & 82    & 48    & 106   &  917 \\
          &       & \textbf{DCSM} & 31    & 30    & 31    & 27    & 30    & 42    & 27    & 32    &  35 \bigstrut[b]\\
\cline{2-12}          & \multirow{6}[2]{*}{\textit{\textbf{nf}}} 
                  & \textbf{Rapidflow} & 1.1s   & 495   & 0.7s   & 154s  & 608   & 719   & 609   & 715   & 1007 \bigstrut[t]\\
          &       & \textbf{CaliG}     & 1.7s   & 639   & 1.2s   & 611s  & 758   & 1247  & 1544  & 969   & 3053 \\
          &       & \textbf{NewSP}     & 18s   & 1183   & 11s   & 48s   & 4387  & 1164  & 1190  & 5083  & 1195 \\
          &       & \textbf{HashIdx} & 0.67 & 749 & 1.1 & 40 & 427 & 215 & 285 & 260 & 185  \\
          &       & \textbf{Edge-Ext}      & 0.4s  & 116   & 0.4s  & 18s   & 149   & 373   & 501   & 837   &  1834 \\
          &       & \textbf{DCSM}      & 0.4s  & 116   & 0.4s  & 18s   & 149   & 110   & 101   & 96    &  128 \bigstrut[b]\\
\cline{2-12}          & \multirow{6}[2]{*}{\textit{\textbf{lb}}} 
                  & \textbf{Rapidflow} & 2.3s  & 2.5s  & 12s   & 455s  & 3.1s  & 2.9s  & 2.5s  & 173s  & 3.1s \bigstrut[t]\\
          &       & \textbf{CaliG} & 9.5s  & 11s   & 449s  & 575s  & 80s   & 713s  & 60s   & 16973s & 961s \\
          &       & \textbf{NewSP} & 22.8s & 5.1s  & 2625s & 497s  & 77s   & 6.3s  & 5.8s  & 9953s & 6.1s \\
          &       & \textbf{HashIdx} & 2.9s & 3.2s & 1.4s & 15s & 2.3s & 1.8s & 0.43s & 26s & 1.5s  \\
          &       & \textbf{Edge-Ext} & 1.3s  & 1.3s  & 1.1s  & 6.2s  & 1.3s  & 2.2s  & 1.1s  & 80s   &  4.2s \\
          &       & \textbf{DCSM} & 1.3s  & 1.3s  & 1.1s  & 6.2s  & 1.3s  & 1.1s  & 0.2s  & 16s   & 0.9s \bigstrut[b]\\
\cline{2-12}          & \multirow{6}[2]{*}{\textit{\textbf{lj}}} 
                  & \textbf{Rapidflow} & 4.8s  & 7.1s  & 17s   & 17s   & 12s   & 92s   & 9.7s  & 101s  & 25s \bigstrut[t]\\
          &       & \textbf{CaliG} & 30s   & 45s   & 265s  & 62s   & 264s  & 1124s & 223s  & 22594s & 5550s \\
          &       & \textbf{NewSP} & 26s   & 17s   & 1284s & 39s   & 333s  & 111s  & 125s  & -     & 5129s \\
          &       & \textbf{HashIdx} & 1.9s & 1.8s & 2.8s & 3.7s & 2.6s & 25s & 1.5s & 29s & 8.9s \\
          &       & \textbf{Edge-Ext} & 1.8s  & 2.9s  & 3.3s  & 3.5s  & 2.2s  & 76s   & 1.9s  & 282s  & 76s  \\
          &       & \textbf{DCSM}     & 0.8s  & 1.0s  & 1.3s  & 1.5s  & 1.2s  & 10s   & 1.4s  & 13s   & 3.2s   \bigstrut[b]\\
    \hline
    \end{tabular}%
  \label{tab:throughput}%
\end{table}%

In this section, we compare the throughput of DCSM with that of other systems. We set the update rate to the maximum (i.e., all updates arrive at time $0$) to measure how long each system takes to process all updates. A shorter processing time indicates that the system has higher throughput.
The graph updater (GU) of DCSM consumes the updates with a batch size of $64$; this setting can ensure the best performance.

\noindent
\subsubsection{Comparison with the CPU Systems}
Table~\ref{tab:throughput} shows the processing time of RapidFlow, CaliG, NewSP, and DCSM for matching query patterns Q1-Q9 on different data graphs. DCSM runs on a single GPU. In most test cases, RapidFlow outperforms the other two CPU-based systems, achieving the best performance on CPU. DCSM further outperforms RapidFlow, achieving speedups ranging from $1.7$x to $42$x, with an average of $10.5$x. The experiment result indicates the effectiveness of our system. The speedup mainly comes from the massive parallelism of the GPU and the various optimizations we proposed. 

For unlabeled matching, on the four graphs—Enron, Amazon, DBLP, and Netflow—DCSM achieves average speedups of $6.2$x, $22$x, $16$x, and $8.6$x over RapidFlow. For labeled matching on the four small graphs—Enron, Amazon, DBLP, and Netflow—DCSM achieves average speedups of $6.7$x, $4.2$x, $5.6$x, and $5.1$x over RapidFlow; on the two large graphs, LSBench and LiveJournal, the average speedup increases significantly to $14.5$x and $13$x. This is because the workload on small graphs is already low, even without labels, and adding labels to both the graphs and query patterns further reduces it. As a result, the search space may be limited to only a few thousand elements—or even just a few hundred. In such cases, warps experience higher overhead when competing for updates, as they may spend more time waiting to acquire the queue lock than performing actual matching.

In some cases, such as az-Q7, nf-Q3, and en-Q2, NewSP and CaliG can slightly outperform RapidFlow; this is because they include optimizations for specific cases, such as the reordering of extensions and operations in NewSP. In most cases, RapidFlow achieves better performance, since the local index used in RapidFlow greatly reduces the search space. Our system does not incorporate RapidFlow's local index method, resulting in a larger search space. Therefore, DCSM does not achieve the order-of-magnitude speedup over RapidFlow comparable to GPU's computational advantage over CPU. Nevertheless, we still outperform RapidFlow through massive parallelism.

\subsubsection{Comparison with the GPU Naive Methods}

As introduced in Section~\ref{sec:exp_setup}, we have two intuitive methods to enable inter-batch parallelism, but both have limitations. In this section, we compare the throughput of DCSM with these two naive methods: Hash-Indexing and Edge-Extension.

From Table~\ref{tab:throughput}, we can see that DCSM consistently outperforms Hash-Indexing. The Hash-Indexing method does not work well for GPU-based subgraph matching tasks, as it causes more thread divergence within a warp during set operations.

Edge-Extension performs especially well on sparse query patterns but is slower on dense query patterns (Q7, Q8, and Q9). This is because the time complexity of Edge-Extension is $O(n^{|E(Q)|})$, which is higher than DCSM’s $O(n^{|V(Q)|})$ on dense query patterns. As a result, Edge-Extension explores a larger search space than DCSM on dense query patterns. However, Edge-Extension still outperforms most CPU-based systems, as the high parallelism of GPUs offsets the drawbacks introduced by its algorithmic complexity.

\subsubsection{Comparison with GCSM}


% Table generated by Excel2LaTeX from sheet 'Sheet4'
\begin{table}[htbp]
  \centering
  \ssmall
  \setlength{\tabcolsep}{4.8pt}
  \caption{Performance comparison between GCSM and DCSM for matching unlabeled and labeled query patterns. The table shows the batch size $b$ used by GCSM. When the batch size is set to $b$, GCSM and DCSM exhibit nearly the same processing time. Both GCSM and DCSM are executed on a single GPU.}
    \begin{tabular}{c|c|c|c|c|c|c|c|c|c|c}
    \hline
          & \textit{\textbf{G}} & \textbf{Q1} & \textbf{Q2} & \textbf{Q3} & \textbf{Q4} & \textbf{Q5} & \textbf{Q6} & \textbf{Q7} & \textbf{Q8} & \textbf{Q9} \bigstrut\\
    \hline
    \hline
    \multirow{4}[8]{*}{\begin{sideways}Unlabeled\end{sideways}} 
                          & \textit{\textbf{en}} & 7456 & 5984 & 7264 & 5216 & 6784 & 5600 & 6304 & 7136 & 5088  \bigstrut\\
    \cline{2-11}          & \textit{\textbf{az}} & 5600 & 6176 & 7136 & 6144 & 6880 & 7232 & 6848 & 6976 & 5632  \bigstrut\\
    \cline{2-11}          & \textit{\textbf{db}} & 6400 & 5824 & 7520 & 6208 & 7200 & 5120 & 7648 & 6496 & 5152  \bigstrut\\
    \cline{2-11}          & \textit{\textbf{nf}} & 6528 & 6176 & 6144 & 4864 & 6400 & 5824 & 4672 & 6272 & 5120  \bigstrut\\
    \hline
    \hline
    \multirow{6}[12]{*}{\begin{sideways}Labeled\end{sideways}} 
                          & \textit{\textbf{en}} & 6976 & 10848 & 6464 & 7712 & 10720 & 7584 & 8384 & 11008 & 8640 \bigstrut\\
    \cline{2-11}          & \textit{\textbf{az}} & 8736 & 7328 & 6816 & 7968 & 8256 & 9888 & 6592 & 9504 & 7232 \bigstrut\\
    \cline{2-11}          & \textit{\textbf{db}} & 9184 & 6752 & 7968 & 9024 & 6688 & 10304 & 9824 & 7424 & 8096 \bigstrut\\
    \cline{2-11}          & \textit{\textbf{nf}} & 8448 & 8832 & 10976 & 9792 & 9632 & 8160 & 8288 & 10080 & 9312 \bigstrut\\
    \cline{2-11}          & \textit{\textbf{lb}} & 6880 & 6144 & 6560 & 5536 & 5088 & 6112 & 5376 & 5056 & 5824 \bigstrut\\
    \cline{2-11}          & \textit{\textbf{lj}} & 4768 & 7520 & 6592 & 5504 & 7072 & 5504 & 6560 & 6816 & 5696 \bigstrut\\
    \hline
    \end{tabular}%
  \label{tab:gcsm_ul}%
\end{table}%


Table~\ref{tab:gcsm_ul} shows the performance comparison between GCSM and DCSM for unlabeled and labeled continuous subgraph matching. As introduced in Section~\ref{sec:Intro}, GCSM processes updates in fixed-size batches, which leads to low GPU utilization when the batch size is small. The graph updater (GU) of DCSM also consumes updates in fixed-size batches. We identify a batch size $b$ at which GCSM and DCSM achieve the same processing time. While GCSM's processing time decreases as $b$ increases, DCSM's execution time remains stable regardless of the batch size, because DCSM can process multiple batches in parallel.

We observe that the numbers in Table~\ref{tab:gcsm_ul} roughly correspond to the number of active warps on the GPU. GCSM achieves the same performance as DCSM when processing with a batch size large enough to saturate GPU resources. For some small labeled data graphs, such as en, az, and db, GCSM requires a larger batch size, because a smaller batch is processed too quickly to effectively hide the kernel launch time for the next batch.


Figure~\ref{fig:scale} shows the trend of GCSM and DCSM performance with the batch size. GCSM's processing time becomes longer on small batch sizes, while DCSM remains stable regardless of the batch size. If all edge updates in the test data are grouped into an extremely large batch, GCSM has almost the same performance as DCSM. We also tested GPU utilization under different batch sizes. We can see that as the batch size decreases, DCSM's GPU utilization remains constant while GCSM's GPU utilization decreases roughly by half successively.


\begin{figure}
    \centering
    \subfloat[]{
        \includegraphics[scale=0.38, page=1]{./fig/gcsm-crop.pdf}
        \label{fig:gcsm}
    }%
    \subfloat[]{
        \includegraphics[scale=0.38, page=1]{./fig/gcsm2-crop.pdf}
        \label{fig:gcsm2}
    }
    \caption{(a) and (b) show the trend of GCSM performance with batch size $b$ on unlabeled db-Q5 and labeled lj-Q5. "All" refers to the batch size being equal to the total number of edge updates in the test data. The numbers marked on the line stand for GPU utilization.}
    \label{fig:scale}
\end{figure}


%______________________________________________________________________________________________________________________________________________________%
\subsection{Response Time}
\begin{figure}[h]
    \centering
    \subfloat[Netflow Q4]{
        \centering
        \includegraphics[scale=0.4, page=1]{./fig/nfq4-crop.pdf}
        \label{fig:rt1}
    }%
    \subfloat[Netflow Q6]{
        \centering
        \includegraphics[scale=0.4, page=1]{./fig/nfq6-crop.pdf}
        \label{fig:rt2}
    }\hfil
    \subfloat[LSBench Q4]{
        \centering
        \includegraphics[scale=0.4, page=1]{./fig/lbq4-crop.pdf}
        \label{fig:rt3}
    }%
    \subfloat[LiveJournal Q9]{
        \centering
        \includegraphics[scale=0.4, page=1]{./fig/ljq9-crop.pdf}
        \label{fig:rt4}
    }\hfil
    \subfloat[LiveJournal Q8]{
        \centering
        \includegraphics[scale=0.4, page=1]{./fig/ljq8-crop.pdf}
        \label{fig:rt5}
    }%
    \subfloat[LSBench Q8]{
        \centering
        \includegraphics[scale=0.4, page=1]{./fig/lbq8-crop.pdf}
        \label{fig:rt6}
    }\hfil
    \caption{Response time over different update rates. The time unit is seconds for Netflow Q4 and milliseconds for all others. GCSM-4096 refers to GCSM configured with a batch size of 4096.}
    \label{fig:response_time}
\end{figure}

In this section, we compared the response times of DCSM, GCSM, and RapidFlow over different update rates (updates / second). 
Figure~\ref{fig:response_time} shows the experimental results.

By observing real-world data, such as Twitter traffic and real-time updates in social networks, we found that in reality, the time intervals between two consecutive arriving updates in dynamic graphs follow an exponential distribution. 
We generated timestamps for the updates in the edge stream for each data graph according to this exponential distribution.
By setting the parameters of the exponential distribution, we can control the time span between the last update and the first update.
A longer time span means fewer updates arrive per unit time, which has a lower $\Delta e_k$ update rate. Conversely, a shorter time span has a higher $\Delta e_k$ update rate.
The horizontal axis in Figure~\ref{fig:response_time} represents update rate. 

The experimental results confirm the effectiveness of the DCSM. 
As shown in Figure~\ref{fig:response_time}, RapidFlow exhibits higher response time at high update rates because the processing capability of a single CPU is limited and cannot guarantee RapidFlow's high throughput. 
At low update rates, GCSM takes longer to form a batch, which also leads to increased response time. 
DCSM can effectively handle various update rates. The parallelism between batches enables DCSM to achieve optimal performance regardless of the update rate. Therefore, DCSM can flexibly adapt to traffic fluctuations over time in real-world scenarios.

We also compared against GCSM-128. However, due to GCSM-128's extremely low throughput, it cannot handle the update rates within the range shown in Figure~\ref{fig:response_time}; the GCSM-128 curve appears as a nearly flat line above RapidMatch. In Figures~\ref{fig:rt1} to~\ref{fig:rt6}, DCSM achieves 70.7x, 53.9x, 60.4x, 80.3x, 87.9x, and 60.4x higher throughput compared to GCSM-128, respectively. GCSM-128's response time drops to near-zero at an extremely low update rate, but GCSM with small batch size performs far worse than CPU systems under typical update rates.


\begin{figure}
    \centering
    \subfloat[Netflow Q4]{
        \includegraphics[scale=0.45, page=1]{./fig/gpuutil-crop.pdf}
        \label{fig:util1}
    }%
    \subfloat[Netflow Q6]{
        \includegraphics[scale=0.45, page=1]{./fig/gpuutil2-crop.pdf}
        \label{fig:util1}
    }
    \caption{GPU utilization vs. update rate (k: thousand, m: million) for different systems}
    \label{fig:gpuutil}
\end{figure}

Figure~\ref{fig:gpuutil} illustrates GPU utilization as a function of the update rate. 
DCSM consistently achieves higher GPU utilization than GCSM across all update rates. 
With large batch sizes, GCSM exhibits low utilization due to its batching strategy, which delays the response time, whereas with small batch sizes, insufficient parallelism limits utilization. 
By contrast, DCSM dynamically schedules incoming updates to available warps, enabling high GPU utilization across update rates. 
At high update rates, the GPU utilization of DCSM, GCSM-500, and GCSM-10000 converges to 100\%.


%______________________________________________________________________________________________________________________________________________________%
\subsection{Overhead Analysis}
\label{sec:overhead}
% Table generated by Excel2LaTeX from sheet 'Sheet7'
\begin{table}[htbp]
  \centering
  \ssmall
  \caption{Time (ms) spent by the graph updater (GU) to synchronously consume all updates.}
    \begin{tabular}{c|c|c|c|c|c|c|c}
    \hline
          & Batch Size & en    & az    & db    & nf    & lb    & lj \bigstrut\\
    \hline
    \hline
    \multirow{5}[2]{*}{\begin{sideways}Unlabeled\end{sideways}} 
          & 1     & 59    & 68    & 114   & 1050  & x     & x \bigstrut[t]\\
          & 32    & 7     & 27    & 11    & 122   & x     & x \\
          & 128   & 5     & 16    & 9     & 70    & x     & x \\
          & 512   & 3     & 3     & 9     & 67    & x     & x \\
          & 2048  & 3     & 3     & 9     & 57    & x     & x \bigstrut[b]\\
    \hline
    \hline
    \multirow{5}[2]{*}{\begin{sideways}Labeled\end{sideways}} 
          & 1     & 11    & 7     & 12    & 210   & 513   & 1155 \bigstrut[t]\\
          & 32    & 7     & 2     & 2     & 24    & 52    & 104 \\
          & 128   & 5     & 2     & 2     & 14    & 41    & 81 \\
          & 512   & 3     & 2     & 2     & 13    & 39    & 76 \\
          & 2048  & 3     & 2     & 2     & 11    & 39    & 77 \bigstrut[b]\\
    \hline
    \end{tabular}%
  \label{tab:gu_time}%
\end{table}%


The MVG data structure causes DCSM to perform more data copying than other systems, which increases memory access overhead.

Table~\ref{tab:gu_time} reports the time required for the graph updater (GU) to process all updates and apply them to the MVG. 
These data were collected while GU, EX, and GC were running concurrently, rather than with GU running in isolation.

By comparing Table~\ref{tab:gu_time} and Table~\ref{tab:throughput}, it can be observed that, in most cases, the time required by GU is significantly lower than that for incremental matching. 
This indicates that GU can produce updates faster than the executor (EX) can consume them, even when updating the MVG with a batch size of 1. 
Thus, EX is the performance bottleneck in most cases.

GU is slower than EX only when DCSM matches size-3 query patterns and GU uses a batch size of 1. 
For example, in Figure~\ref{tab:throughput}, lj-Q1 requires 800 ms for matching, while GU takes 1155 ms; nf-Q5 requires 800 ms, while GU takes 1050 ms.
However, in DCSM, the batch size of GU is typically set to 64 or larger. Therefore, at high update rates, GU does not become a performance bottleneck. At low update rates, timeouts may cause GU to update the MVG with a batch size of 1, but this millisecond-level slowdown does not noticeably increase response time or affect user experience.

We also profiled DCSM's memory footprint. At high update rates, GPU memory consumption rises sharply within a few milliseconds and then decreases gradually at a rate comparable to that of EX, as GC operates at a speed determined by EX, which is much faster than GU. At update rates equal to or lower than the processing speed of EX, GPU memory consumption remains stable over time.


% 在一些轻量级的query任务上,比如lj-Q1, nf-Q5
%______________________________________________________________________________________________________________________________________________________%
\subsection{Ablation Study}
\begin{figure}
    \centering
    \subfloat[Batch Size = $64$]{
        \includegraphics[scale=0.45, page=1]{./fig/speedups-crop.pdf}
        \label{fig:bs1}
    } \\
    \subfloat[Batch Size = $128$]{
        \includegraphics[scale=0.45, page=1]{./fig/speedups2-crop.pdf}
        \label{fig:bs2}
    } 
    \caption{Speedups brought by parallel graph updater on different data graphs. Naive refers to a single warp sequentially copying different arrays, rather than copying multiple arrays in parallel.}
    \label{fig:async_gu_data}
\end{figure}


Figure~\ref{fig:async_gu_data} shows the speedup achieved by the parallel graph updater (GU) across different data graphs. 

We observe that assigning more warps to the GU achieves better speedups, as more warps can issue more in-flight memory request instructions in parallel within the same cycle, thereby saturating the GPU global memory bandwidth. 
Higher parallelism ensures that memory throughput is not bounded by instruction dispatch.

However, when the number of warps continues to increase such that the total number of threads exceeds the batch size, further increasing warps no longer brings significant speedup. 
For example, in Figure~\ref{fig:bs1}, using 8 warps produces almost the same speedup as using 4 warps.
This is because different batches cannot be updated to the MVG in parallel, and the average neighbor size in the data graph is very small, resulting in the total length of all arrays to be copied being smaller than the number of available threads.

We also observe that a single warp achieves the most significant speedup compared to Naive, where one warp iterates 32 times.
Since most neighbor arrays have a size of only 1-2, a single warp in parallel only needs to iterate 2 to 3 times. Single warp parallelization greatly reduces the number of iterations and increases memory request parallelism.
Meanwhile, 2 warps only halve the number of iterations compared to 1 warp.

Our proposed pipeline processing method also brings significant speedup to the GU, and becomes particularly effective under high update rates. The copy operation of the subsequent batch can hide the sort operation of the pr

Through these optimizations, the GU can process updates at high speed, thereby avoiding performance bottlenecks that would be slower than EX for most queries.
%______________________________________________________________________________________________________________________________________________________%

\section{Related Works}
Continuous subgraph matching (CSM) originates from static subgraph matching. The design of static subgraph matching systems has a significant impact on CSM.
There are many efforts to design high performance systems for subgraph matching. 
These include CPU systems, such as GraphZero \cite{mawhirter2019graphzero}, Dryadic \cite{mawhirter2021dryadic}, and GraphPi\cite{shi2020graphpi}, Peregrine \cite{jamshidi2020peregrine}, Automine \cite{mawhirter2019automine}, G2Miner \cite{chen2022efficient},  Sandslash \cite{chen2021sandslash}, DecoMine \cite{chen2022decomine}, and RapidMatch \cite{sun2020rapidmatch}
These also include  GPU systems, such as Gunrock \cite{wang2020fast}, GPSM \cite{tran2015fast}, GSI \cite{zeng2020gsi}, cuTS \cite{xiang2021cuts}, PBE \cite{guo2020gpu}, and STMatch \cite{wei2022stmatch}.  These systems introduce various program optimization methods to improve hardware utilization. 

Continuous subgraph matching has been studied for decades on CPU. IncIsoMatch \cite{fan2013incremental} is the first continuous subgraph matching work, it retrieves the graph region affected by the edge update of the query and performs the static subgraph matching again in the region to get the incremental results.
Graphflow \cite{kankanamge2017graphflow} eliminates the repeated matching in IncIsoMatch for each edge update and derives the multi-way join equations for batched continuous graph matching, performing incremental matching based on the equations. SJ-Tree \cite{choudhury2015selectivity} builds an index tree for computing incremental matching results. It stores partial results to eliminate redundant computation that can be precomputed before runtime, but the index tree leads to memory explosion when processing large graphs.  TurboFlux \cite{kim2018turboflux} proposes a new data structure called data-centric graph structure that can reduce the storage amount of partial results to get a better tradeoff between memory consumption and computation overhead. SymBi \cite{min2021symmetric} is the successive work of TurboFlux and proposes a pruning method for candidate vertices with query edges. Rapidflow \cite{sun2022rapidflow} builds a local index for breaking the matching order fixation and a dual matching elimination to reduce the redundant computation caused by query automorphisms. CaliG \cite{yang2023fast} divides the query into two parts, it computes the partial results with the first part and gets the final result with the second part, the extension in the second part can be finished without backtracking. NewSP \cite{li2024newsp} gets the best performance on the CPU. NewSP avoids premature expansions of the search space by postponing expansion at the operation level.

In recent years, some studies \cite{wei2024gcsm, qiu2024gpu} have started using GPUs to accelerate continuous subgraph matching. GCSM \cite{wei2024gcsm} designs a sampling algorithm for memory optimization, it only places frequently accessed vertices of graphs on the GPU to support extremely large graphs that exceed the GPU memory size. GAMMA \cite{qiu2024gpu} is batch-dynamic subgraph matching with warp-level work stealing and coalesced search for reducing redundant computations. Both GCSM and GAMMA are offline systems, based on the assumption that we can form an extremely large batch size.


\section{Conclusion}
This paper presents the first continuous subgraph matching system on GPU capable of different batches in parallel. It offers a highly practical solution for various real-world scenarios. However, DCSM also introduces new challenges, such as memory inefficiency due to volatile memory and deeply hidden redundant computations that could be eliminated during preprocessing. These issues are not present in previous systems. Many previous works cache a large number of intermediate results to avoid recomputing incremental matching from scratch. A new challenge arises in how to manage these intermediate results efficiently on GPUs and how to make the indexing structures used to store them compatible with our system. There is still significant room for optimizing DCSM in terms of both performance and efficiency. 



\bibliographystyle{ACM-Reference-Format}
\bibliography{jp}
\end{document}
\endinput
%%
%% End of file `sample-sigconf.tex'.
